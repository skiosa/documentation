%!TEX root = ../dokumentation.tex

\chapter{Projektziel}

Betrachtet man weltweit die aktuell größten IT-Firmen und Start-ups, so fällt schnell auf, dass diese überwiegend auf dem gleichen Konzept aufgebaut sind.
Egal ob Amazon, Spotify oder Netflix, besteht das zentrale Angebot dieser Firmen aus einer Plattform, die als Anlaufstelle für ein vorherig schwer zu navigierendes Medium genutzt wird.
Bei Spotify findet man an einer Stelle alle Lieder, die man hören wollen könnte und das Auffinden von neuen Genren und Musikern wird einem erheblich erleichtert.
Netflix und Amazon haben ähnliche Konzepte was Filme und (physische) Einkaufswaren angeht.
Ziel dieses Projektes soll es sein eine solche Plattform für RSS Feeds und den darin übermittelten Artikeln zu erstellen.

\section{Projektanforderungen}
Im folgenden soll nun beschrieben werden, wie zu Anfang des Projektes die Requirements aufgefasst und beschreiben wurden.
Aus technischen Gründen hat sich im Verlauf des Prozesses herausgestellt, dass einige dieser Anforderungen nicht implementierbar sind und wurden daher abgeändert.
Mehr Informationen hierzu sind im Abschnitt \ref{tech_changes} nachzulesen.

\subsection{Must-have Requirements}

\subsection{Erreichbare Ziele}

\subsection{Mögliche weiteren Ziele}

\section{Anfängliche Software Architektur}
\todo{Services etc beschreiben}
\todo{Software Auswahl beschreiben (express, angular)}
