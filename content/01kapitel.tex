%!TEX root = ../dokumentation.tex

\chapter{Projektziel}

Betrachtet man weltweit die aktuell größten IT-Firmen und Start-ups, so fällt schnell auf, dass diese überwiegend auf dem gleichen Konzept aufgebaut sind.
Egal ob Amazon, Spotify oder Netflix, besteht das zentrale Angebot dieser Firmen aus einer Plattform, die als Anlaufstelle für ein vorherig schwer zu navigierendes Medium genutzt wird.
Bei Spotify findet man an einer Stelle alle Lieder, die man hören wollen könnte und das Auffinden von neuen Genren und Musikern wird einem erheblich erleichtert.
Netflix und Amazon haben ähnliche Konzepte was Filme und (physische) Einkaufswaren angeht.
Ziel dieses Projektes soll es sein eine solche Plattform für RSS Feeds und den darin übermittelten Artikeln zu erstellen.
Das Projekt ``Skiosa'' soll eine Plattform anbieten, die zur zentralen Anlaufstelle für RSS Feeds agieren kann so ihren Nutzern das Entdecken von neuen Artikeln aus bekannten, sowie unbekannten Feeds erleichtert.

\section{Projektanforderungen}
Im folgenden soll nun beschrieben werden, wie zu Anfang des Projektes die Requirements aufgefasst und beschreiben wurden.
Aus technischen Gründen hat sich im Verlauf des Prozesses herausgestellt, dass einige dieser Anforderungen nicht so implementierbar sind und wurden daher leich abgeändert.
Mehr Informationen hierzu sind im Abschnitt \ref{tech_changes} nachzulesen.

\subsection{Unverzichtbare Ziele}
Zu Begin müssen zentrale Interaktionsmechanismen für Artikel und Feeds gegeben sein.
Seiten für das Einsehen von Artikeln und Feeds bilden daher den Kern der Appikation.
% NOTE: INHALT wurde rausgestrichen und warum muss in section \ref{tech_changes} erklärt werden
Auf dessen Einsichtsseiten muss es möglich sein Basisinformationen, wie etwa Titel, Inhalt oder Beschreibung, nachlesen zu können.
Bei Artikeln muss die Einsichtsseite so auf den eigentlichen Artikel verweisen, während Feeds auf die in ihnen enthaltenen Artikel verlinken.

Den größten Mehrwert der Plattform sollen vorgeschlagene Artikel bilden.
Es soll Nutzern möglich sein innerhalb der Plattform neue Artikel zu entdecken.
Funktional bedeutet dies, dass die wichtigste Anforderung an die Plattform eine Seite ist, auf welcher Vorschläge für Artikel abgebildet sind, welche sich ständig erneuern.

Damit Skiosa zu einer zentralen Anlaufstelle für RSS-Feeds werden kann, müssen Nutzer einen Grund haben um wiederholt auf die Plattform zugreifen zu wollen.
Subscription- und Bookmark-Funktionalitäten für eingeloggte Nutzer können genau diese Lücke füllen.
Hieraus leiten sich drei weitere funktionale Requirements ab;
Erstens muss die Plattform ein User-Management besizen, in welcher sich Nutzer einloggen und registrieren können.
Zweitens soll es angemeldeten Nutzern der Plattform möglich sein, bestimmte Feeds zu abbonieren und diese, sowie deren neusten Artikeln, im nachhinein aufzurufen zu können.
Drittens sollen angemeldeten Nutzer Artikel ``bookmarken'' (dt. vermerken) können und diese zu einem späteren Zeitpunkt in der Plattform wieder finden.

Auf nicht-funktionaler Ebene ist es für die Plattform wichtig, dass diese aesthetisch gut aussieht.
Konkrete nicht-funktionale Requirements die sich hierraus ableiten sind, dass die Oberflächen der Plattform einer festen Design Sprache folgen und User Eperience Design Seitenübergreifen einheitlich gestaltet ist.

\subsection{Erreichbare Ziele}
Zusätzlich für das Projekt Hilfreiche Funktionalitäten, die zu Anfang als erreichbar angesehen wurden wenden sich um das Optimieren der Vorgeschlagenen Artikeln.
Es wäre für Nutzer, die es aus bspw. Spotify oder YouTube gewohnt sind ihre Preferenzen als ``Likes'' anzugeben, hilfreich, wenn sie dies ebenfalls in Skiosa machen könnten.
Als optionales Requirement folgt hierraus, dass Nutzer bei Artikeln durch einen ``Like''-Knopf angeben können, welche Artikel Ihnen besonders gefallen und dann ihnen mehr Artikel vorgeschlagen werden, die zu diesen ähnliche sind.

Zusätzlich ist es bei Artikeln von Zeitschriften und Blogs üblich diese zu einzukategoriesieren.
In Skiosa wäre es hilfreich, wenn ein Benutzer diese Kategorien angeben könnten und die Vorschläge für diese Nutzer sich dann an den angegebenen Kategorien orrientiert.

\subsection{Mögliche weiteren Ziele}
Eine weitere mögliche Erweiterung der Plattform, die aber im Zeitrahmen des Semesters nicht realistisch erscheint ist das erstellen einer Content Management Seite.
In dieser Content-Management Seite könnten besitzer von RSS feeds ihre eigenen Feeds hinzufügen und Verwalten.
Hier wäre es möglich Polling-Intervalle, Aussehen, Beschreibungen oder Titel von Feeds abzändern.
Zusätzlich wäre es hierbei notwendig Appeal Prozesse für falsch zugewisene Besitzer anzubieten.

\section{Anfängliche Software Architektur}
\todo{Services etc beschreiben}
\todo{Software Auswahl beschreiben (express, angular)}
