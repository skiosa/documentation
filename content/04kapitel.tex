% !TeX root = ../dokumentation.tex

\chapter{Sprints} \label{sprints}

\section{Sprint 1}

\subsection{Ziele Sprint 1}
Ziel des ersten Sprints war die ersten Schritte im PI \enquote{Initial Infrastructure} (siehe Abschnitt \ref{initial_infra}) zu gehen.
Hierbei ging es darum die Arbeitsgrundlage für zukünftige Sprints zu legen.
Konkret war es also das Ziel zum einen die grundlegende Infrastruktur aufzusetzen (Server, Domain zu kaufen und aufzusetzen), sowie die eingesetzten Kollaborationstools, wie Jira, Confluence, Github und Figma zu konfigurieren.
Zum anderen aber auch die Grundarchitektur zu klären und erste Entwürfe in Figma zu erstellen.

\subsection{Probleme und Verbesserungen}
Im Retro des ersten Sprints wurde das \enquote{Start, Stop, Continue} Verfahren gewählt.
Hierbei ist aufgefallen, dass es Probleme mit zu vielen Blocking Tickets, aufgeblähten Diskussionsrunden und einem verfehlten Sprint Ziel gab. 
Als Konsequenz wurde hierraus gezogen, dass zusätzlich zu den Weeklys nun ein Bi-Weekly Meeting existiert. Sprint Kapazitäten sollen vor den jeweiligen Sprints geschätzt werden (Die Auswirkung dieser Änderung ist in Abschnitt \ref{orga_changes} nachzulesen).

\subsection{Bearbeitete User Storys}
\subsubsection{SKIOS-22 POC for keycloak}
Das Ticket \enquote{SKIOS-22 POC for keycloak} hatte folgende User Story:
\begin{quotation}
    As a developer I would like to have a POC for two services using keycloak~\parencite{web/Keycloak} so that I`ll know how users will be managed in the future. \\
    \textbf{Acceptance Criteria}
    \begin{itemize}
        \item A backend service has an api that can get information about the logged in user
        \item A frontend service can create users (in any primitive way) and log in + send a request to the aformentioned endpoint
        \item Our keycloak~\parencite{git/skiosa/orm} instance is configured to work with both services
    \end{itemize}
\end{quotation}
Bearbeitet von Tim Horlacher.
Reviewed von Lukas Huida.

\subsubsection{SKIOS-25 LaTeX Template}
Das Ticket \enquote{SKIOS-25 LaTeX Template} hatte folgende User Story:
\begin{quotation}
    As a product owner I want a working LaTeX Template to be present so that the platform is easy to document.
\textbf{Acceptance Criteria}
\begin{itemize}
    \item The template is compiling to pdf
    \item The template includes boilerplate information such as title, our names, the date, etc.
    \item Potentially unnessicary pages (such as the \enquote*{Sperrvermerk}) are NOT in the template
    \item A mechanism for dealing with sources (ex. Citavi, BibTex, etc.) is clarified
\end{itemize}
\end{quotation}
Bearbeitet von Jonas Eppard.
Reviewed von Lukas Huida.

\subsubsection{SKIOS-26 Preliminary Recommendation Service}
Das Ticket \enquote{SKIOS-26 Preliminary Recommendation Service} hatte folgende User Story:
\begin{quotation}
    As a developer I want a mock/boilerplate recommendation service to be created so that development can start.
    \textbf{Acceptance Criteria}
    \begin{itemize}
        \item Boilerplate for a basic service internal architecture is created
        \item Mock endpoints for sending random recommendations not coming from a database are created
        \item Basic testing framework is set up for this service
    \end{itemize}
\end{quotation}
Bearbeitet von Jonas Eppard.
Reviewed von Tim Horlacher.

\subsubsection{SKIOS-28 Clarify Article Structure}
Das Ticket \enquote{SKIOS-28 Clarify Article Structure} hatte folgende User Story:
\begin{quotation}
    As a product owner I want a definition for the article Structure inside TypeScript and GraphQL.
\textbf{Acceptance Criteria}
\begin{itemize}
    \item A Structure inside TypeScript is created
    \item The Structure uses GraphQL Decorators
\end{itemize}
\end{quotation}
Bearbeitet von Jonas Eppard und Tim Horlacher.
Reviewed von Tim Horlacher und Lukas Huida.

\subsubsection{SKIOS-29 Undesigned Angular Boilerplate}
Das Ticket \enquote{SKIOS-29 Undesigned Angular Boilerplate} hatte folgende User Story:
\begin{quotation}
    As a developer I would like a boilerplate for angular to be created so that I can start work on the project. \\
    \textbf{Acceptance Criteria}
    \begin{itemize}
        \item An angular boilderplate exists with basic pages and routing
        \item A central structure for requests (using services) is created
        \item Infrastructure for passing global data (such as a username) is implemented
    \end{itemize}
\end{quotation}
Bearbeitet von Tim Horlacher.
Reviewed von Lukas Huida.

\subsubsection{SKIOS-31 Evaluate Core-Service}
Das Ticket \enquote{SKIOS-31 Evaluate Core-Service} hatte folgende User Story:
\begin{quotation}
    As a developer I would like to know which ORM will be used and whether or not to base our APIs around GraphQL~\parencite{web/GraphQL} or to stick with REST within the Core service. \\
    \textbf{Acceptance Criteria}
    \begin{itemize}
        \item It is clear, what ORM to use and is documented somewhere (ex. comments of this story)
        \item It is clear if GraphQL~\parencite{web/GraphQL} is used and findings are documented somewhere aswell
    \end{itemize}
\end{quotation}
Bearbeitet von Tim Horlacher.
Reviewed von Lukas Huida.

\subsubsection{SKIOS-32 Preliminary Polling service}
Das Ticket \enquote{SKIOS-32 Preliminary Polling service} hatte folgende User Story:
\begin{quotation}
    As a developer I would like a basic boilerplate for the polling service so that development can start.
    Functionally this Polling service only loads mock data into the database (intervall will be implemnted in k8s and not here) and isn't a \enquote{basic polling service}.
\textbf{Acceptance Criteria}
\begin{itemize}
    \item Mock data is loaded into the database at startup
    \item Basic boilerplate and infrastructure for the service is created (ex. project and file/directory structure)
    \item Mock data isn't provided from an RSS feed 
    \item Basic testing framework is set up
\end{itemize}
\end{quotation}
Bearbeitet von Jannik Springer.
Reviewed von Lukas Huida.

\subsubsection{SKIOS-34 List of Participants}
Das Ticket \enquote{SKIOS-34 List of Participants} hatte folgende User Story:
\begin{quotation}
    As our lecturer I would like a list of group members so that I can grade the project. \\
    \textbf{Acceptance Criteria}
    \begin{itemize}
        \item A list is created including all 11 names and matnr.
        \item Said list is given to the professor
    \end{itemize}
\end{quotation}
Bearbeitet von Tim Horlacher.
Reviewed von Lukas Huida.

\subsubsection{SKIOS-52 Devcontainer Guide}
Das Ticket \enquote{SKIOS-52 Devcontainer Guide} hatte folgende User Story:
\begin{quotation}
    As a developer I would like a guide on the usage of devcontainers for development. \\
    \textbf{Acceptance Criteria}
    \begin{itemize}
        \item A guide is created in confluence
        \item A sample devcontainer is implemented
        \item Other developers are taught about devcontainers 
    \end{itemize}
\end{quotation}
Bearbeitet von Tim Horlacher.
Reviewed von Lukas Huida.

\subsubsection{SKIOS-53 Devcontainer for LaTeX Documentation}
Das Ticket \enquote{SKIOS-53 Devcontainer for LaTeX Documentation} hatte folgende User Story:
\begin{quotation}
    As a developer I want a devcontainer for the documentation
\textbf{Acceptance Criteria}
\begin{itemize}
    \item A docker devcontainer is created
    \item The devcontainer is compatible with \ac{VSC}'s Extension Remote - Containers
    \item \LaTeX is running in devcontainer
\end{itemize}
\end{quotation}
Bearbeitet von Jonas Eppard.
Reviewed von Lukas Huida.

\subsubsection{SKIOS-13 Create Basic Infrastructure}
Das Ticket \enquote{SKIOS-13 Create Basic Infrastructure} hatte folgende User Story:
\begin{quotation}
    Als Product Owner möchte ich die Möglichkeit haben das unser Produkt weltweit sicher erreichbar ist. Um das zu erreichen, muss ein Server mit einer passenden gemietet werden. \\
    \textbf{Acceptance Criteria}
    \begin{itemize}
        \item Die Domain \enquote{Skiosa.de} bestellt und dem Server zugewiesen.
        \item Einen passenden Server bestellt und mit notwendigen Tools wie SSH aufgesetzt.
        \item K3S Basis Installation.
        \item kubectl von Remote erreichbar und eingerichtet.
        \item Firewall Regeln angelegt um nur Web, SSH und Kubectl zuzulassen.
    \end{itemize} 
\end{quotation}
Bearbeitet von Lukas Huida.
Reviewed von Tim Horlacher.

\subsubsection{SKIOS-18 Basic Kubernetes Infra}
Das Ticket \enquote{SKIOS-18 Basic Kubernetes Infra} hatte folgende User Story:
\begin{quotation}
    As a infra member I want a basic Infrastructure based on Kubernetes. This Infrastructure should container some CI/CD pipeline software and examples to implement
    a CI/CD pipeline for our services. \\
    \textbf{Acceptance Criteria}
    \begin{itemize}
        \item Working Ingress-Controller with correct Certs
        \item Self-Hosted Docker-Registry
        \item ArgoCD Setup
        \item Drone.io Setup
        \item SonarQube Setup
        \item Keycloak Setup
        \item HashiCorp Vault Setup (Optional)
    \end{itemize}
\end{quotation}
Bearbeitet von Lukas Huida.
Reviewed von Tim Horlacher.

\subsubsection{SKIOS-9 Create Continous Integration Pipeline}
Das Ticket \enquote{SKIOS-9 Create Continuos Integration Pipeline} hatte folgende User Story:
\begin{quotation}
    As a developer I want my code to be tested and analyzed by sonar after every new commit so that our platform can ensure stability. \\
    \textbf{Acceptance Criteria}
    \begin{itemize}
        \item A pipeline exists for every service.
        \item A commit pipeline is triggered after every commit on a non master branch. This pipeline runs through: unit tests and sonarqube.
        \item A pr pipeline is triggered after a pull requests is created. This pipeline runs through: unit tests, integration tests and sonarqube.
        \item A main pipeline is triggered after a merge of a master branch. This pipeline runs through: unit tests, container build, CD-Script and sonarqube.
    \end{itemize}
\end{quotation}
Bearbeitet von Lukas Huida.
Reviewed von Tim Horlacher.

\subsubsection{SKIOS-10 Create Continous Delivery Pipeline}
Das Ticket \enquote{SKIOS-10 Create Continuos Delivery Pipeline} hatte folgende User Story:
\begin{quotation}
    As a developer I want my code to be delivered to prod after I merge it to master. \\
    \textbf{Acceptance Criteria}
    \begin{itemize}
        \item pipeline is triggered after merge to master
        \item pipeline tests code with unit/sonar before merge (is already managed by ci-pipeline)
        \item pipeline tests code with newman after merge → fail leads to rollback
    \end{itemize}
\end{quotation}
Bearbeitet von Lukas Huida.
Reviewed von Jonas Eppard.

\subsubsection{SKIOS-21 Initialize Keycloak}
Das Ticket \enquote{SKIOS-21 Initialize Keycloak} hatte folgende User Story:
\begin{quotation}
    As a developer I would like to have Keycloak setup so that we can test its usefulness. \\
    \textbf{Acceptance Criteria}
    \begin{itemize}
        \item A Keycloak server is running in our cluster and is reachable.
    \end{itemize}
\end{quotation}
Bearbeitet von Lukas Huida.
Reviewed von Tim Horlacher.

\subsubsection{SKIOS-30 Preliminary Core service}
Das Ticket \enquote{SKIOS-30 Preliminary Core service} hatte folgende User Story:
\begin{quotation}
    As a developer I want to have a basic core service with some boilerplate and mock endpoints to get started in the frontend. \\
    \textbf{Acceptance Criteria}
    \begin{itemize}
        \item be able to fetch mock articles with graphql
    \end{itemize}
\end{quotation}
Bearbeitet von Lukas Huida.
Reviewed von Tim Horlacher.

\subsubsection{SKIOS-37 Decision of Programming Database}
Das Ticket \enquote{SKIOS-37 Decision of Programming Database} hatte folgende User Story:
\begin{quotation}
    As a developer I need to know which database is used, to know which database connector for ORMs are needed. \\
    \textbf{Acceptance Criteria}
    \begin{itemize}
        \item A decision of a database is done
        \item The decision is documented in Confluence
    \end{itemize}
\end{quotation}
Bearbeitet von Lukas Huida.
Reviewed von Jonas Eppard.

\subsubsection{SKIOS-38 Contribution Guidelines}
Das Ticket \enquote{SKIOS-38 Contribution Guidelines} hatte folgende User Story:
\begin{quotation}
    As a developer I would like to have a guidline for making contributions so that we have a defined process to get code resulting from stories into production. \\
\textbf{Acceptance Criteria}
\begin{itemize}
    \item Contribution Guidlines are documented in Confluence
    \item Git Guidelines for naming schemes for git commits, branch names etc., squash/rebase merging are documented
    \item Review and Testing has a process, defining what should be reviewed (integration tests, code style, etc) and who should review it
    \item Code comments and code styles are defined and agreed upon by every member
    \item Documentation style, including what to document where (Confluence vs. Git) is clarified
    \item It is clarified when a task counts as done (review of dev, review of po, dev and tech lead, only review of tech/service lead)
\end{itemize}
\end{quotation}
Bearbeitet von Simon Morgenstern.
Reviewed von Amos Groß.

\subsubsection{SKIOS-39 Guidelines Testing}
Das Ticket \enquote{SKIOS-39 Guidelines Testing} hatte folgende User Story:
\begin{quotation}
    As a developer I would like to have a guideline for testing so that I can know if I'm testing my code correctly. \\
\textbf{Acceptance Criteria}
\begin{itemize}
    \item The types of tests are clarified (unit, integration, manual, cypress/selenium, etc.)
    \item The amount of test that have to be written is clear (ex. test coverage, integration test for ever use case or error, etc)
    \item A decision is made when, where and by whom tests have to be written (ex. in separate stories per epic, implicitly with every story, etc.)
    \item All findings are documented in confluence
\end{itemize}
\end{quotation}
Bearbeitet von Simon Morgenstern.
Reviewed von Amos Groß.

\subsubsection{SKIOS-40 Testing Frameworks}
Das Ticket \enquote{SKIOS-40 Testing Frameworks} hatte folgende User Story:
\begin{quotation}
    As a developer, I would like to know which testing frameworks I need to use so that I can write tests. \\
\textbf{Acceptance Criteria}
\begin{itemize}
    \item Testing frameworks for every language are decided upon
    \item The testing frameworks are added to the services documentation in confluence
    \item Programs to to integration tests with are clear
\end{itemize}
\end{quotation}
Bearbeitet von Simon Morgenstern.
Reviewed von Amos Groß.

\subsubsection{SKIOS-49 Create Example Service}
Das Ticket \enquote{SKIOS-49 Create Example Service} hatte folgende User Story:
\begin{quotation}
    As a developer I want an example service to test the CI/CD pipeline. \\
    \textbf{Acceptance Criteria}
    \begin{itemize}
        \item An example service exists
        \item A example deployment is created
        \item A example CI pipeline is working with the example service
    \end{itemize}
\end{quotation}
Bearbeitet von Lukas Huida.
Reviewed von Tim Horlacher.

\subsubsection{SKIOS-50 Deploy Database}
Das Ticket \enquote{SKIOS-50 Deploy Database} hatte folgende User Story:
\begin{quotation}
    As a developer I want a central database to store my objects and articles. \\
    \textbf{Acceptance Criteria}
    \begin{itemize}
        \item PostgreSQL is deployed in a db namespace
        \item db is reachable in the default namespace with credentials stored at the vault
    \end{itemize}
\end{quotation}
Bearbeitet von Lukas Huida.
Reviewed von Jonas Eppard.

\subsubsection{SKIOS-64 Guideline for Issues}
Das Ticket \enquote{SKIOS-64 Guideline for Issues} hatte folgende User Story:
\begin{quotation}
    As a product owner, I would like to have a guideline for new issues so that the team can understand what their work is. \\
\textbf{Acceptance Criteria}
\begin{itemize}
    \item Issue Guideline is documented in confluence
    \item Structure for future stories is clear
\end{itemize}
\end{quotation}
Bearbeitet von Simon Morgenstern.
Reviewed von Amos Groß.

\subsubsection{SKIOS-23 Design Logo}
Das Ticket \enquote{SKIOS-23 Design Logo} hatte folgende User Story:
\begin{quotation}
    As a user i would like to see a logo when visiting Skiosa. \\
    \textbf{Acceptance Criteria}
    \begin{itemize}
        \item A good looking icon exists
    \end{itemize}
\end{quotation}
Bearbeitet von Sabrina Ladner.
Reviewed von Lukas Huida.

\subsubsection{SKIOS-55 Skiosa Colorscheme}
Das Ticket \enquote{SKIOS-55 Skiosa Colorscheme} hatte folgende User Story:
\begin{quotation}
    As a ui/ux designer, i would like to know which colors to use for mockups. \\
    \textbf{Ideas for a possible solution}
    \begin{itemize}
    \item Look at colors already used by the figma mockups 
    \item Modify Logo or colors (ex. dark blue to dark green)
    \item Possibly use a color scheme generator as suggested by Simon Morgenstern
    \end{itemize}
    \textbf{Acceptance Criteria}
    \begin{itemize}
    \item A colorscheme has been created
    \item Colorscheme is decided on and briefly documented under “UI/UX” in confluence
    \item Colorscheme has light and dark mode
    \item Ladner Sabrina's Logo now reflects this color scheme
    \end{itemize}
\end{quotation}
Bearbeitet von Sabrina Ladner.
Reviewed von Amos Groß.

\subsubsection{SKIOS-11 Update Mockups for possible design language}
Das Ticket \enquote{SKIOS-11 Update Mockups for possible design language} hatte folgende User Story:
\begin{quotation}
    As a ui/ux designer, I would like to like to have a design language to follow when creating mockups.
    If you're unsure about what a design langage is, the following article explains it:  (this story references the \enquote{Style guidelines} part of their definition)
    Also interesting thing to look up might be UI/UX design in general to understand why design languages matter and are necessary for good software. \\
\textbf{Acceptance Criteria}
\begin{itemize}
    \item deas are documented in confluence
    \item Figma Mock ups now represent new design language
    \item Decisions on fonts, separation of pages and page structure are made
    \item A clear design language is implemented for Skiosa
\end{itemize}
\end{quotation}
Bearbeitet von Amos Groß.
Reviewed von Lukas Huida.

\subsubsection{SKIOS-27 Document Initial Requirements}
Das Ticket \enquote{SKIOS-27 Document Initial Requirements} hatte folgende User Story:
\begin{quotation}
    As a product owner I want the plattform's requirements and features to be documented, so that we can refer to it in the final documentation. \\
\textbf{Acceptance Criteria}
\begin{itemize}
    \item All Requirements are listed in a confluence page
    \item Must haves and optional requirements are listed or indicated as such
\end{itemize}
\end{quotation}
Bearbeitet von Amos Groß.
Reviewed von Lukas Huida.

\subsubsection{SKIOS-33 Mockups for Central Pages}
Das Ticket \enquote{SKIOS-33 Mockups for Central Pages} hatte folgende User Story:
\begin{quotation}
    As a developer I would like to have mockups of the main pages, so that it is clear how to design the frontend. \\
\textbf{Acceptance Criteria}
\begin{itemize}
    \item Mockups are created in figma
    \item Overview and View pages have mockups
    \item The Mockups have been discussed with the team (ie. review)
    \item Mockups contain desktop and mobile versions
\end{itemize}
\end{quotation}
Bearbeitet von Amos Groß.
Reviewed von Lukas Huida.

\subsubsection{SKIOS-43 Fix Jira Workflow}
Das Ticket \enquote{SKIOS-43 Fix Jira Workflow} hatte folgende User Story:
\begin{quotation}
    As a developer I want to see if a Issue/Story/Ticket is Done. At the moment if its done the story is not crossed out as at should be. \\ 
\textbf{Acceptance Criteria}
\begin{itemize}
    \item Tickets/Story/Issues are crosses out if there are done/resolved.
\end{itemize}
\end{quotation}
Bearbeitet von Amos Groß.
Reviewed von Lukas Huida.

\subsubsection{SKIOS-57 Figma Colors}
Das Ticket \enquote{SKIOS-57 Figma Colors} hatte folgende User Story:
\begin{quotation}
    As a ui/ux designer, I would like to have a central style in figma to base color choices off of.

    Subject of this story is to create a style based off of whatever the current color scheme is and to replace all instances of that color with the style. \\

\textbf{Acceptance Criteria}
\begin{itemize}
    \item A new style is created in the figma draft
    \item All hardcoded colors are replaced with mentions of styles
    \item The colors in the style have names (as explained above)
\end{itemize}
\end{quotation}
Bearbeitet von Marcel Alex.
Reviewed von Amos Groß.

\subsubsection{SKIOS-66 LaTeX Introduction}
Das Ticket \enquote{SKIOS-66 LaTeX Introduction} hatte folgende User Story:
\begin{quotation}
    As a member of the team, I would like to have an introduction to LaTeX, so that I can help with documentation.
\textbf{Acceptance Criteria}
\begin{itemize}
    \item LaTeX was presented to the team and team members understand gist of working with LaTeX
    \item README of LaTeX Project summarises the basics needed to write sections, etc.
\end{itemize}
\end{quotation}
Bearbeitet von Jannik Springer und Jonas Eppard.
Reviewed von Lukas Huida.

\subsubsection{Skios-35 Decision of Programming Language of Recomodation Engine}
Das Ticket \enquote{Skios-35 Decision of Programming Language of Recomodation Engine} hatte folgende User Story:
\begin{quotation}
    As a developer I need to know in which Programming Language the recomondation Engine is written. \\
    \textbf{Acceptance Criteria}
    \begin{itemize}
        \item A decision of a Programming Language is done
        \item The decision is documented in Confluence
    \end{itemize}    
\end{quotation}
Bearbeitet von Theo Krinitz
Reviewed von Tim Horlacher.

\subsubsection{Skios-36 Decision of Programming Language of Polling Service}
Das Ticket \enquote{Skios-36 Decision of Programming Language of Polling Service} hatte folgende User Story:
\begin{quotation}
    As a developer I need to know in which programming language the polling-service is written. \\
    \textbf{Acceptance Criteria}
    \begin{itemize}
        \item A decision of a programming language is done
        \item The decision is documented in Confluence
    \end{itemize}
\end{quotation}
Bearbeitet von Theo Krinitz
Reviewed von Tim Horlacher.

\subsubsection{SKIOS-58 Approachable Filter}
Das Ticket \enquote{SKIOS-58 Approachable Filter} hatte folgende User Story:
\begin{quotation}
    As a developer, I would like a filter for approachable tickets to be created, so that I can pull tickets with greater ease.\\
    \textbf{Acceptance Criteria}
    \begin{itemize}
        \item A filter exists at the top of the sprint board
        \item That filter filters for approachable tickets
    \end{itemize}
\end{quotation}
Bearbeitet von Theo Krinitz
Reviewed von Lukas Huida.


\subsubsection{SKIOS-59 Design Settings Interface}
Das Ticket \enquote{SKIOS-59 Design Settings Interface} hatte folgende User Story:
\begin{quotation}
    As a frontend developer, I would like to know how our settings interface should look so that I can start creating it.
    \textbf{The Mockup page should include:}
        \begin{itemize}
            \item A way to view and change the necessary settings
            \item An indication of what user is logged in
            \item A saving mechanism
        \end{itemize}
\end{quotation}
\textbf{Acceptance Criteria}
    \begin{itemize}
        \item Figma Mockups are designed for the settings page
        \item Mockups use existing components for articles, buttons and input fields
        \item Design uses SKIOSA colorscheme
        \item Design includes ways to change
        \begin{itemize}
            \item colorscheme
            \item language
            \item interest in categories
        \end{itemize}
        \item Design includes a link to changing password or to logout 
    \end{itemize}
Bearbeitet von Sophie Gösch.
Reviewed by Simon Morgenstern.

\subsubsection{SKIOS-60 User Experience for Likes and Bookmarks}
Das Ticket \enquote{SKIOS-60 User Experience for Likes and Bookmarks} hatte folgende User Story:
\begin{quotation}
    As a ui/ux designer, I would like to know what a possible workflow for working with likes and bookmarks would be so that we can start implementing it.

    Possibly look up what User Experience Design is before working on this ticket \url{https://xd.adobe.com/ideas/career-tips/what-is-ux-design/}
\end{quotation}
\textbf{Acceptance Criteria}
    \begin{itemize}
        \item Definitions and Decisions are documented in confluence
        \item It is clear which pages to create for Bookmarks and Likes
        \item User Interaction with the side bar is clarified and a decision is made on where to place like or bookmark buttons
    \end{itemize}
\textbf{Ideas for a possible solution}
    \begin{itemize}
        \item Define Bookmarks in terms of our application (for reference see \href{https://skiosa.atlassian.net/wiki/spaces/SKIOSA/pages/1015809/Draft+for+Project+Architecture}{Project Architecture} or \href{https://skiosa.atlassian.net/wiki/spaces/SKIOSA/pages/4063267/Epic+Drafts}{Epic Drafts})
        \item Document findings in confluence under “UI/UX”
        \item Clarify how we should show users their bookmarks (in the side bar, in a sperate page), if a seperate page is chosen, think about where to place it (top level, nested under overview, nested under user pages, etc)
        \item Clarify if we need some type of read or watched like in youtube
        \item Clarify if Bookmarks work as marking of read bookmarks or of articles that you want to read in the future (and how this would impact how to structure bookmarks  ~ alphabetically, by unseen, let the recomendation engine use ML, etc)
        \item Should Likes also have an overview and how would they differ from Bookmarks
        \item Possibly decide if we even need bookmarks and likes or if one of them would be enough
        \item In Summary: define the where, what and how of bookmarks and likes
    \end{itemize}
Bearbeitet von Sophie Gösch.
Reviewed by Amos Groß.

\subsubsection{SKIOS-61 Design Subscription Interface}
Das Ticket \enquote{SKIOS-61 Design Subscription Interface} hatte folgende User Story:
\begin{quotation}
    As a frontend developer, I would like to know how our subscription interface should look so that I can start creating it.
    \textbf{The Mockup page should include:}
        \begin{itemize}
            \item A list of subscribed feeds (either up top or in seperate frame)
            \item A button or method to unsubscribe
            \item A list of most recent articles
            \item All other things that might be helpfull for a subscription page
        \end{itemize}
\end{quotation}
\textbf{Ideas for a possible solution}
    \begin{itemize}
        \item Use colorscheme and guidelines (see linked issues) to create pages
        \item For more information on the requirements for our user page see: \href{https://skiosa.atlassian.net/wiki/spaces/SKIOSA/pages/4030475/Initial+Requirements}{Initial Requirements} (might not be all too helpful in this case though)
        \item Possibly copy this page from youtube
    \end{itemize}
\textbf{Acceptance Criteria}
    \begin{itemize}
        \item Figma Mockups include subscription interface
        \item Interface includes the components named above
    \end{itemize}
Bearbeitet von Sophie Gösch.
Reviewed by Simon Morgenstern.

/\subsubsection{SKIOS-62 Design Feed overview}
Das Ticket \enquote{SKIOS-62 Design Feed overview} hatte folgende User Story:
\begin{quotation}
    As a frontend developer, I would like to know how our feed overview interface should look so that I can start creating it.

    The feed overview is like the channel page in YouTube (ex. \url{https://www.youtube.com/channel/UC9m7D4XKPJqTPCLSBym3BCg/featured} ), just for RSS feeds.
    \textbf{The Mockup page should include:}
        \begin{itemize}
            \item Name of feed
            \item subscription status
            \item Recent articles
            \item possibly more information on feed
        \end{itemize}
\end{quotation}
\textbf{Ideas for a possible solution}
    \begin{itemize}
        \item Use colorscheme and guidelines (see linked issues) to create pages
        \item For more information on the requirements for our user page see: \href{https://skiosa.atlassian.net/wiki/spaces/SKIOSA/pages/4030475/Initial+Requirements}{Initial Requirements} (might not be all too helpful in this case though)
        \item Possibly copy this page from youtube
    \end{itemize}
\textbf{Acceptance Criteria}
    \begin{itemize}
        \item Figma Mockups include feed overview interface
        \item Interface includes the components named above
    \end{itemize}
Bearbeitet von Sophie Gösch.
Reviewed by Amos Groß.

\section{Sprint 2}

\subsection{Ziele Sprint 2}
Ziel des zweiten Sprints war es das verfehlte Sprint Ziel des ersten Sprints nachzuholen und somit das erste \ac{PI} fertig zu stellen. Zusätzlich sollten nun erste Backend Endpunkte für das zweite \ac{PI} \enquote{Article Views and Base Homepage} erstellt werden.
Ein weiteres Ziel war die Service Boilerplates und die dafür notwendige \ac{CI/CD} Infrastruktur zu erstellen. Mit diesen Boilerplates sollen die Endpoints für Artikel View Pages und die Homepage erstellt werden.
Für die Benutzeroberflächen, welche im folgenden Sprint eingefügt werden, sollen Mockups erstellt werden. 

\subsection{Probleme und Verbesserungen}
In der Retrospektive des zweiten Sprints wurde die Teamarbeit spielerisch, mithilfe einer Piratenanalogie, analysiert.
Probleme die aufgefallen sind: Tickets waren häufig nicht allgemein verständlich beschrieben, es wurde zu wenig für die Abgabe Dokumentation gemacht wurde und zu viel Arbeit in letzter Minute vor Sprintende erledigt.
Positiv aufgefallen ist dass das Sprintziel wesentlich besser erreicht wurde und es viel Fortschritt und weniger Disskussionsbedarf gab.
Als Fazit wurden Deadlines für Stories erstellt und bei zukünftigen Estimation Meetings sollten die Stories vom Service Lead, statt vom Product Owner, vorgestellt werden. Letzteres stellt sicher dass Stories vom Service Lead verstanden werden bevor sie bearbeitet werden sollen. 

\subsection{Bearbeitete User Storys}

\subsubsection{SKIOS-67 Enhanced Git/Jira Automation}
Das Ticket \enquote{SKIOS-67 Enhanced Git/Jira Automation} hatte folgende User Story:
\begin{quotation}
    As a developer I would like enhanced automation for Jira with the following requirements
    \begin{itemize}
        \item Dev starts new branch for his work, story changes status to → in progress
        \item Dev starts PR for his Story → story changes status to → needs review
        \item PR is merged → story changes status to → done
        \item Sync reviewer between Jira and Git (optional) 
    \end{itemize}

    \textbf{Acceptance Criteria}
    \begin{itemize}
        \item Requirements above are covered
        \item functionality works for all git-repositories
    \end{itemize}
\end{quotation}
Bearbeitet von Lukas Huida.
Reviewed von Tim Horlacher.

\subsubsection{SKIOS-69 Preliminary ORM}
Das Ticket \enquote{SKIOS-69 Preliminary ORM} hatte folgende User Story:
\begin{quotation}
    Als Programmierer will ich eine Datendefinition in einer Datenbank
    haben, und diese als Packet einbinden können.
\end{quotation}
Diese wurde folgendermaßen gelöst:
\begin{quotation}
Die Datendefinition wurde durch TypeORM~\parencite{web/TypeORM} dargestellt.
Diese wurde in dem Repository skiosa/orm~\parencite{git/skiosa/orm} als NPM-Package~\parencite{web/npm} abgebildet.
Es wurden die Article von RSS-Feeds in dem Schema aus Abbildung~\ref{fig:databaseORM} abgebildet.
\begin{figure}
    \includegraphics[width=\linewidth]{Database_Model.png}
    \caption{Datendefinition innerhalb der Datenbank}
    \label{fig:databaseORM}
\end{figure}
\end{quotation}
Bearbeitet von Jonas Eppard und Tim Horlacher.

\subsubsection{SKIOS-83 List of initial Feeds}
Das Ticket \enquote{SKIOS-83 List of initial Feeds} hatte folgende User Story:
\begin{quotation}
    As a developer I would like a list of initial feeds that we can poll before giving users the ability to add them in manually. 
    Goal of this story is to find 20-30 good (non reddit) feeds.\\
    \textbf{Acceptance Criteria}
    \begin{itemize}
        \item 20-30 feeds have been found and contain actual articles
        \item Structure of feeds is similar enough for polling service 
        \item The List is documented either in confluence or JSON in polling-service
    \end{itemize}
\end{quotation}
Bearbeitet von Theo Krinitz
Reviewed von Jannik Springer

\subsubsection{SKIOS-92 Initialize Keycloak}
Das Ticket \enquote{SKIOS-92 Initialize Keycloak} hatte folgende User Story:
\begin{quotation}
    As a developer I want to connect keycloak~\parencite{web/Keycloak} to our services and protect certain resources. \\
    \textbf{Acceptance Criteria}
    \begin{itemize}
        \item Two keycloak realms exist (production, testing)
        \item Our frontend connects to keycloak and resources are protectable
        \item Our backend connects to keycloak and resources are protectable
        \item Both realms have roles for users and admins
    \end{itemize}
\end{quotation}
Bearbeitet von Tim Horlacher.
Reviewed von Lukas Huida.

\subsubsection{SKIOS-93 Document Submission Guideline}
Das Ticket \enquote{SKIOS-93 Document Submission Guideline} hatte folgende User Story:
\begin{quotation}
    As a team member i would like to know what should be documented in our final documentation. \\
    \textbf{Acceptance Criteria}
    \begin{itemize}
        \item A confluence page should document what should be documented
    \end{itemize}
\end{quotation}
Bearbeitet von Tim Horlacher.
Reviewed von Amos Groß.

\subsubsection{SKIOS-74 Extend workflow to include wontfix}
Das Ticket \enquote{SKIOS-74 Extend workflow to include wontfix} hatte folgende User Story:
\begin{quotation}
    As a team member, I would like a wontfix state to disregard tickets.
    Goal of this story is to modify the current issue workflow used by jira.
    Note: this might require admin rights (contact PO or Architect for this) \\
    \textbf{Acceptance Criteria}
    \begin{itemize}
        \item Wontfix status exists
        \item The status contains the done property (is striked out)
    \end{itemize}
\end{quotation}
Bearbeitet von Lukas Huida.
Reviewed von Amos Groß.

\subsubsection{SKIOS-85 Repeated Polling runs}
Das Ticket \enquote{SKIOS-85 Repeated Polling runs} hatte folgende User Story:
\begin{quotation}
    As a developer I would like the polling service to be run as a cron job. \\
    \textbf{Acceptance Criteria}
    \begin{itemize}
        \item Polling service is called in Intervalls.
        \item Pod is not running all the time.
    \end{itemize}
\end{quotation}
Bearbeitet von Lukas Huida.
Reviewed von Tim Horlacher.

\subsubsection{SKIOS-55 Skiosa Colorscheme}
Das Ticket \enquote{SKIOS-55 Skiosa Colorscheme} hatte folgende User Story:
\begin{quotation}
    As a ui/ux designer, i would like to know which colors to use for mockups. \\
    \textbf{Ideas for a possible solution}
    \begin{itemize}
    \item Look at colors already used by the figma mockups 
    \item Modify Logo or colors (ex. dark blue to dark green)
    \item Possibly use a color scheme generator as suggested by Simon Morgenstern
    \end{itemize}
    \textbf{Acceptance Criteria}
    \begin{itemize}
    \item A colorscheme has been created
    \item Colorscheme is decided on and briefly documented under “UI/UX” in confluence
    \item Colorscheme has light and dark mode
    \item Ladner Sabrina's Logo now reflects this color scheme
    \end{itemize}
\end{quotation}
Bearbeitet von Sabrina Ladner.
Reviewed von Amos Groß.

\subsubsection{SKIOS-83 Create Issue Template}
Das Ticket \enquote{SKIOS-83 Create Issue Template} hatte folgende User Story:
\begin{quotation}
    As a member of the Jira org, I would like to have a default template for issues
\end{quotation}
Bearbeitet von Amos Groß.
Reviewed von Lukas Huida.

\subsubsection{SKIOS-63 Create Issue Template}
Das Ticket \enquote{SKIOS-63 Create Issue Template} hatte folgende User Story:
\begin{quotation}
    As a member of the Jira org, I would like to have a default template for issues. \\
\textbf{Acceptance Criteria}
\begin{itemize}
    \item An issue template exists for User Stories
\end{itemize}
\end{quotation}
Bearbeitet von Amos Groß.
Reviewed von Lukas Huida.

\subsubsection{SKIOS-77 Subscription Management Endpoints}
Das Ticket \enquote{SKIOS-77 Subscription Management Endpoints} hatte folgende User Story:
\begin{quotation}
    As a front end developer I would like endpoints for subscription management so that I can create its interface.
    Endpoint need to fulfill the following use cases:
    \begin{itemize}
        \item List subscribed feeds of current user
        \item Paginated fetch of most recent articles from all feeds
        \item Subscribe current user to feed
    \end{itemize}

    This functionality is part of the core-service \\

\textbf{Acceptance Criteria}
\begin{itemize}
    \item Use cases above can be performed with created endpoints
    \item Implementation contains reusable components for rest of plattform (aka is good code)
\end{itemize}
\end{quotation}
Bearbeitet von Amos Groß.
Reviewed von Tim Horlacher.

\subsubsection{SKIOS-106 ORM in Polling Service}
Das Ticket \enquote{SKIOS-106 ORM in Polling Service} hatte folgende User Story:
\begin{quotation}
    As a developer of our polling service I would like to use our shared ORM.
    Goal of this story is to replace all current ORM or datatbase logic with the new common node module and to load mock articles into the database following the structures provided.
\textbf{Acceptance Criteria}
\begin{itemize}
    \item Polling service now uses the common ORM logic
    \item Database is loaded up with min. of 5-10 Articles (using ORM)
\end{itemize}
\end{quotation}
Bearbeitet von Jannik Springer.
Reviewed von Lukas Huida.

\subsubsection{SKIOS-81 creating feed service}
Das Ticket \enquote{SKIOS-81 creating feed service} hatte folgende User Story:
\begin{quotation}
As a front-end developer I would like to have service to base feed views off of so that i can create their pages.
The created service should cover the following use cases:
\begin{itemize}
    \item fetch single feeds
    \item fetch articles of feed (paginated)
\end{itemize}
This functionality is part of the core-service.\\
\textbf{Acceptance Criteria}
\begin{itemize}
    \item Acceptance Criteria
    \item This functionality is implemented in core-service
\end{itemize}
\end{quotation}
Bearbeitet von Marcle Alex.
Reviewed von Lukas Huida.

\section{Sprint 3}
Ziel des dritten Sprints war es die Frontend Komponenten des \ac{PI} \enquote{Article Views and Base Homepage} fertig zustellen und die Backend Funktionalität für den nächsten \ac{PI} \enquote{User Interaction} vorzubereiten.
Im Frontend soll die Feed Overview Page, die Article View Page sowie die Startseite erstellt werden.
Wie im letzten Sprint sollen Mockups für den nachfolgenden Sprint erstellt werden.
Ein weiterer Fokus ist die Dokumentation der vergangenen und des aktuellen Sprints.

\subsection{Probleme und Verbesserungen}
In der Retrospektive des dritten Sprints wurde ein neues Tool getestet. Da dieses Meeting online stattfand ermöglichte das Tool trotz der fehlenden Gemeinschaft einen effektiven Austausch.
Positiv aufgefallen ist dass in diesem Sprint sehr erfolgreich viele Stories bearbeitet werden konnten und es zwischen Teammitgliedern weniger Reibungen gab. Negativ aufgefallen ist dass Reviews noch nicht gründlich genug waren. Als Lösungsvorschlag wurde hier eingebracht dass mehr Zeit für Reviews eingeplant werden soll.

\subsection{Bearbeitete User Storys}

\subsubsection{SKIOS-116 Structure and table of contents for submission (\LaTeX)}
Das Ticket \enquote{SKIOS-116 Structure and table of contents for submission (\LaTeX)}
hatte folgende User Story:
\begin{quotation}
    As a team member, I would like to have a rough structure to orient myself while writing our submission documentation.\\
    For this story, please read the requirements and guidelines set out by Garidis and develop a rough idea on how to structure our \LaTeX project.\\
    \textbf{Acceptance Criteria}
    \begin{itemize}
        \item Table of contents is created (with \textbackslash{}section, \textbackslash{}subsection, etc.) in \LaTeX
        \item Structure reflects guidelines of Garidis
        \item Structure is explained in confluence page
        \item Existing \LaTeX~stories have a defined place where their pages will go
    \end{itemize}
\end{quotation}
Dies wurde folgendermaßen gelöst:
\begin{quotation}
    Es wurde die Struktur dieses \LaTeX-Dokuments angelegt. Hierbei musste nur das Inhaltsverzeichnis
    angelegt werden, da das \LaTeX-Template schon vorhanden war.
    Die Verwendung wurde in Confluence dokumentiert.
\end{quotation}
Bearbeitet von Jonas Eppard.

\subsubsection{Skios-79 Basic Article Endpoints}
Das Ticket \enquote{Skios-79 Basic Article Endpoints} hatte folgende User Story:
\begin{quotation}
    As a front-end developer I would like to have endpoints to base article views off of so that i can create their pages.\\
    \textbf{Acceptance Criteria}
    \begin{itemize}
        \item Endpoints are created to cover the case above including fieldresolver for 1-n and n-m relations
        \item This functionality is implemented in core-service
    \end{itemize}
\end{quotation}
Bearbeitet von Theo Krinitz
Reviewed von Jonas Eppard

\subsubsection{SKIOS-134 Fix core-service tests}
Das Ticket \enquote{SKIOS-134 Fix core-service tests} hatte folgende User Story:
\begin{quotation}
    As a developer I want a working testing framework, which implements unit and integration tests.\\
    \textbf{Acceptance Criteria}
    \begin{itemize}
        \item newman is installed inside dev container
        \item npm script is added that runs newman tests
    \end{itemize}
\end{quotation}
Bearbeitet von Theo Krinitz
Reviewed von Jannik Springer

\subsubsection{SKIOS-73 Place new logo in git, jira, confluence, etc}
Das Ticket \enquote{SKIOS-73 Place new logo in git, Jira, confluence, etc} hatte folgende User Story:
\begin{quotation}
    As a Product Owner I want our logo to be placed in the git repository, Jira \& confluence. \\
    \textbf{Requirements}
    \begin{itemize}
        \item Add branding for Skiosa
        \item Possibly also change look of Jira (at the top) and make it use our colorscheme.  
    \end{itemize}   
    
    \textbf{Acceptance Criteria}
    \begin{itemize}
        \item Skiosa Brandig is present in Git and Jira.
    \end{itemize}
\end{quotation}
Bearbeitet von Lukas Huida.
Reviewed von Jonas Eppard.

\subsubsection{SKIOS-76 Automatic Reviewer Suggestions}
Das Ticket \enquote{SKIOS-76 Automatic Reviewer Suggestions} hatte folgende User Story:
\begin{quotation}
    As a developer I want to have a way to automatically suggest reviewers for a PR. \\
    \textbf{Acceptance Criteria}
    \begin{itemize}
        \item Added Codeowners to every repo.
    \end{itemize}
\end{quotation}
Bearbeitet von Lukas Huida.
Reviewed von Tim Horlacher.

\subsubsection{SKIOS-91 Checklist for PRs}
Das Ticket \enquote{SKIOS-91 Checklist for PRs} hatte folgende User Story:
\begin{quotation}
    As a developer, I would like to have a reminder to check off the checklist from our contributing guidelines. \\
    \textbf{Acceptance Criteria}
    \begin{itemize}
        \item Every PR contains comment or default text including the checklist.
    \end{itemize}
\end{quotation}
Bearbeitet von Lukas Huida.
Reviewed von Jonas Eppard.

\subsubsection{SKIOS-111 Angular Design Boilerplate}
Das Ticket \enquote{SKIOS-111 Angular Design Boilerplate} hatte folgende User Story:
\begin{quotation}
    As a frontend developer I would like our frontend to have the basic components.
    Goal of this story ist to create components for:
    \begin{itemize}
        \item Buttons
        \item The Sidebar
        \item feeds
    \end{itemize}
\textbf{Acceptance Criteria}
\begin{itemize}
    \item Angular components are created for Buttons, Sidebar, Feeds
    \item The components are visibly identical to those in figma
    \item Button components can execute code when clicked on
    \item components should look according to current color mode (light / dark)
    \item Favicon of frontend is Skiosa Favicon
    \item Sidebar expands on mobile (see hamburger icon on pages like github, etc.)
\end{itemize}
\end{quotation}
Bearbeitet von Simon Morgenstern.
Reviewed von Lukas Huida.

\subsubsection{SKIOS-119 Frontend README}
Das Ticket \enquote{SKIOS-119 Frontend README} hatte folgende User Story:
\begin{quotation}
    As a developer, I would like a README in frontend service so that I read up on information on how to use it. \\
\textbf{Acceptance Criteria}
\begin{itemize}
    \item README is created
    \item README includes:
    \begin{itemize}
        \item development guide (how to start application, use devcontainer)
        \item description of service
        \item requirements for service
        \item service lead
    \end{itemize}
\end{itemize}
\end{quotation}
Bearbeitet von Lukas Huida.
Reviewed von Amos Groß.

\subsubsection{SKIOS-120 Deployment README}
Das Ticket \enquote{SKIOS-120 Deployment README} hatte folgende User Story:
\begin{quotation}
    As a developer, I would like a README in deployment so that I read up on information on how to use it. \\
\textbf{Acceptance Criteria}
\begin{itemize}
    \item README is created
    \item README includes:
    \begin{itemize}
        \item development guide (how to start application, use devcontainer)
        \item description of service
        \item requirements for service
        \item service lead
    \end{itemize}
\end{itemize}
\end{quotation}
Bearbeitet von Lukas Huida.
Reviewed von Amos Groß.

\subsubsection{SKIOS-137 Angular Service Framework}
Das Ticket \enquote{SKIOS-137 Angular Service Framework} hatte folgende User Story:
\begin{quotation}
    As a developer I would like to have angular services for our qraphql endpoints. \\
\textbf{Acceptance Criteria}
\begin{itemize}
    \item A library is found to automatically generate or facilitate graphql query creation
    \item Framework for creating graphql queries is created
    \item existing services are modified
    \item at minimum, one service is created to explain how to use this framework
\end{itemize}
\end{quotation}
Bearbeitet von Lukas Huida.
Reviewed von Amos Groß.

\subsubsection{SKIOS-138 Use Keycloak in GraphQL}
Das Ticket \enquote{SKIOS-138 Use Keycloak in GraphQL} hatte folgende User Story:
\begin{quotation}
    As a developer I want to secure some GraphQL resolver with our Keycloak. \\
    \textbf{Acceptance Criteria}
    \begin{itemize}
        \item Every resolver can be secured with Keycloak.
        \item An example resolver is created with Keycloak authentication.
    \end{itemize}
\end{quotation}
Bearbeitet von Lukas Huida.
Reviewed von Tim Horlacher.

\subsubsection{SKIOS-123 LaTeX Requirements Page}
Das Ticket \enquote{SKIOS-123 LaTeX Requirements Page} hatte folgende User Story:
\begin{quotation}
    As a team member I would like to have our requirements documented in confluence so that I won't fail this class. \\
\textbf{Acceptance Criteria}
\begin{itemize}
    \item A page is created in LaTeX
    \item Page includes:
    \begin{itemize}
        \item functional requirements
        \item non functional requirements 
        \item optional and mandatory
    \end{itemize}
    \item Has a table explaining changes made to these over the course of the project
\end{itemize}
\end{quotation}
Bearbeitet von Amos Groß.
Reviewed von Lukas Huida.

\subsubsection{SKIOS-133 ORM Readme}
Das Ticket \enquote{SKIOS-133 ORM Readme} hatte folgende User Story:
\begin{quotation}
    As a developer, I would like a README for ORM so that I read up on information on how to use it. \\
\textbf{Acceptance Criteria}
\begin{itemize}
    \item README is created
    \item README includes:
    \begin{itemize}
        \item development guide (how to create tags and to embed it to package.json)
        \item description of what this module does
        \item code owners 
    \end{itemize}
\end{itemize}
\end{quotation}
Bearbeitet von Jonas Eppard.
Reviewed von Lukas Huida.

\subsubsection{SKIOS-122 LaTeX Architecture Overview}
Das Ticket \enquote{SKIOS-122 LaTeX Architecture Overview} hatte folgende User Story:
\begin{quotation}
    As a team member I would like to have our architecture documented in confluence so that I won’t fail this class.
\textbf{Acceptance Criteria}
\begin{itemize}
    \item A page is created in our LaTeX documentation
    \item Services are listed and explained
    \item CI/CD infrasturture is explained
    \item Security Infrastructure (Keycloak)
    \item The Page explains the architecture in text form (prosa) 
    \item A diagram (not UML) is created to visualize our plattform architecture
\end{itemize}
\end{quotation}
Bearbeitet von Tim Horlacher und Lukas Huida.
Reviewed von Amos Groß.

\subsubsection{SKIOS-84 Initial Uniform Polling Implementation}
Das Ticket \enquote{SKIOS-84 Initial Uniform Polling Implementation} hatte folgende User Story:
\begin{quotation}
    As a developer I would like an initial implementaton of the polling service.\\
    This implementation should parse a single \enquote{type} of RSS feeds including at minimum the following:
    
    \begin{itemize}
        \item Link
        \item Titel
        \item Beschreibung
    \end{itemize}

    If an article doesn't contain this information it is ignored.\\
    If an article contains more information than this that information will also be ignored (will be parsed in later story)\\
    After parsing, it loads the information into the database. \\
\textbf{Acceptance Criteria}
\begin{itemize}
    \item Service can poll the structure above
    \item Polling service loads data into DB
    \item Multiple runs of the service don't produce duplicate data
\end{itemize}
\end{quotation}
Bearbeitet von Jannik Springer.
Reviewed von Lukas Huida.

\subsubsection{SKIOS-109 Backend Feed Mutation}
Das Ticket \enquote{SKIOS-109 Backend Feed Mutation} hatte folgende User Story:
\begin{quotation}
As a developer, I would like to have muations for creating and deleting feeds.
Goal of this story is to create Graphql Mutation to add a feed into our database.\\
\textbf{Acceptance Criteria}
\begin{itemize}
    \item Graphql mutation is created for feeds
    \item calling one of the mutations results in a feed being created in the database
    \item calling the mutation is possible only knowing the url of a feed (all other “not null” fields are left blank or filled with defaults)
    \item calling the other mutation (by ID) results in the specified feed being deleted
    \item deletion should only be possible for logged in users with a specific keycloak role
    \item creation should be possible for all logged in users
\end{itemize}
\end{quotation}
Bearbeitet von Marcel Alex und Tim Horlacher.
Reviewed von Lukas Huida.

\subsubsection{SKIOS-115 Core Service README}
Das Ticket \enquote{SKIOS-115 Core Service README} hatte folgende User Story:
\begin{quotation}
    As a developer, I would like a README in core service so that I read up on information on how to use it. \\
\textbf{Acceptance Criteria}
\begin{itemize}
    \item README is created
    \item README includes:
    \begin{itemize}
        \item development guide (how to start application, use devcontainer)
        \item description of what this module does
        \item requirements for service
        \item service lead
    \end{itemize}
\end{itemize}
\end{quotation}
Bearbeitet von Tim Horlacher
Reviewed von Jonas Eppard.

\subsubsection{SKIOS-117 Polling Service README}
Das Ticket \enquote{SKIOS-117 Polling Service README} hatte folgende User Story:
\begin{quotation}
    As a developer, I would like a README in polling service so that I read up on information on how to use it. \\
\textbf{Acceptance Criteria}
\begin{itemize}
    \item README is created
    \item README includes:
    \begin{itemize}
        \item development guide (how to start application, use devcontainer)
        \item description of service
        \item requirements for service
        \item service lead
    \end{itemize}
\end{itemize}
\end{quotation}
Bearbeitet von Jannik Springer.
Reviewed von Lukas Huida.

\subsubsection{SKIOS-121 PoC READMEs}
Das Ticket \enquote{SKIOS-121 PoC READMEs} hatte folgende User Story:
\begin{quotation}
    As a developer, I would like a README in all PoCs so that I read up on information on how to use it. \\
\textbf{Acceptance Criteria}
\begin{itemize}
    \item README is created
    \item README includes:
    \begin{itemize}
        \item development guide (how to start application, use devcontainer)
        \item goal of PoC
        \item link to Story of POC
        \item result of PoC
    \end{itemize}
\end{itemize}
\end{quotation}
Bearbeitet von Tim Horlacher
Reviewed von Amos Groß.

\subsubsection{SKIOS-124 \LaTeX Use cases page and UML}
Das Ticket \enquote{SKIOS-124 \LaTeX Use cases page and UML} hatte folgende User Story:
\begin{quotation}
    As a team member I would like to have our use cases documented in LaTeX so that I won't fail this class. \\
\textbf{Acceptance Criteria}
\begin{itemize}
    \item A page is created in LaTeX
    \item Page includes a use case diagram for:
    \begin{itemize}
        \item overview page
        \item settings page
        \item view page
    \end{itemize}
    \item The different use cases for SKIOSA are explained in text form (prosa) and embedd and explain the use case diagrams
    \item Page is spellchecked and contains no spelling mistakes (TEST THIS IN REVIEW)
    \item Text has been read by at least one other team member
\end{itemize}
\end{quotation}
Bearbeitet von Jonas Eppard und Sabrina Ladner.
Reviewed von Sabrina Ladner und Amos Groß.

\subsubsection{SKIOS-127 Finalize ORM}
Das Ticket \enquote{SKIOS-127 Finalize ORM} hatte folgende User Story:
\begin{quotation}
    As a developer, I would like to finalize our ORM, so that I can load data into the DB without having to fear it breaking.
\textbf{Acceptance Criteria}
\begin{itemize}
    \item Urls are Unique
    \item Articles have a timestamp
    \item Changes are signed off by both Architects
\end{itemize}
\end{quotation}
Bearbeitet von Jonas Eppard und Jannik Springer.
Reviewed von Tim Horlacher, Lukas Huida und Jannik Springer.

\subsubsection{SKIOS-82 Feed Overview Page}
Das Ticket \enquote{SKIOS-82 Feed Overview Page} hatte folgende User Story:
\begin{quotation}
    As a user of Skiosa, I would like to view a feed, so that I can decide if I want to subscribe to it.
    Content of this story is to implemented an overview page for feeds based on our mockups (this will exclude recommended feeds if part of the mockup). \\
\textbf{Acceptance Criteria}
\begin{itemize}
    \item The page displays: 
    \begin{itemize}
        \item name
        \item description
        \item articles
    \end{itemize}
    \item Page is designed as specified in figma
    \item The subscribe button subscribes a user (if not logged in redirects to login page)
\end{itemize}
\end{quotation}
Bearbeitet von Marcel Alex.
Reviewed von Lukas Huida.

\subsubsection{SKIOS-80 Article View Page}
Das Ticket \enquote{SKIOS-80 Article View Page} hatte folgende User Story:
\begin{quotation}
    As a user of Skiosa, I would like to view an article.
    Content of this story is to implemented the view page based on our mockups without its references to other articles or likes/bookmarks. \\
\textbf{Acceptance Criteria}
\begin{itemize}
    \item Title, link, description, etc. are displayed 
    \item Page uses endpoints from core-sevice to achieve this
\end{itemize}
\end{quotation}
Bearbeitet von Simon Morgenstern, Jonas Eppard.
Reviewed von Lukas Huida.

\subsubsection{SKIOS-105 Structure and table of contents for submission LaTeX}
Das Ticket \enquote{SKIOS-105 Structure and table of contents for submission LaTeX} hatte folgende User Story:
\begin{quotation}
    As a team member, I would like to have a rough structure to orient myself while writing our submission documentation.
    For this story, please read the requirements and guidelines set out by Garidis and develop a rough idea on how to structure our LaTeX project.
\textbf{Acceptance Criteria}
\begin{itemize}
    \item Table of contents is created (with \\section, \\subsection, etc.) in LaTeX
    \item Structure reflects guidelines of Garidis
    \item Structure is explained in confluence page
    \item Existing LaTeX stories have a defined place where their pages will go
\end{itemize}
\end{quotation}
Bearbeitet von Jonas Eppard.
Reviewed von Amos Groß.

\subsubsection{SKIOS-110 Mockup for Adding Feeds}
Das Ticket \enquote{SKIOS-110 Mockup for Adding Feeds} hatte folgende User Story:
\begin{quotation}
    As a developer I would like a figma design for an adding feeds page. \\
\textbf{Acceptance Criteria}
    \begin{itemize}
        \item Figma page follows color scheme and design guidelines
        \item User has a field to add a URL
        \begin{itemize}
            \item Disabled Name, Description field.
        \end{itemize}
        \item Sidebar is updated to include button for link to this page
        \item Workflow for adding link includes:
        \begin{itemize}
            \item Testing the url / rss feed if it can be parsed
            \item Parse Name, Description and TTL
            \item Name and Description fields are updated to the parsed ones, user can adjust them after parsing
            \item (Add Button will send the actual post request)
        \end{itemize}
    \end{itemize}
\end{quotation}
Bearbeitet von Sabrina Ladner.
Reviewed von Amos Groß.

\subsubsection{SKIOS-78 Design Bookmarks Page}
Das Ticket \enquote{SKIOS-78 Design Bookmarks Page} hatte folgende User Story:
\begin{quotation}
    As a frontend developer, I would like to know how our bookmark page should look. \\
\textbf{Acceptance Criteria}
    \begin{itemize}
        \item Figma Mockups are designed for the bookmarks page
        \item Mockups use existing components for articles, buttons and input fields
        \item Pages follow workflow for likes and bookmarks
        \item Design uses SKIOSA colorscheme
    \end{itemize}
\end{quotation}
Bearbeitet von Sabrina Ladner.
Reviewed von Amos Groß.

\subsubsection{SKIOS-71 Design Login and registration pages}
Das Ticket \enquote{SKIOS-71 Design Login and registration pages} hatte folgende User Story:
\begin{quotation}
    As a frontend developer, I would like to know how our login page and registration page should look. \\
    This page should \textbf{not} include the sidebar, but be designed as a page for keycloak.
    \textbf{Acceptance Criteria}
    \begin{itemize}
        \item Figma Mockups are created for the login and registration pages
        \item Mockups include all fields and elements (such as a field to enter name, username, email, etc) present in Keycloak
        \item Design does not have a sidebar
        \item Design uses colors of colorscheme
        \item Design uses common input field or button components
    \end{itemize}
\end{quotation}
Bearbeitet von Sabrina Ladner.
Reviewed von Amos Groß.

\subsubsection{SKIOS-56 Common Figma Components}
Das Ticket \enquote{SKIOS-56 Common Figma Components} hatte folgende User Story:
\begin{quotation}
    As a ui/ux designer, I would like to have common components to drop in for Mockups. \\
    These components include:
    \begin{itemize}
        \item Buttons
        \item Input fields
        \item Main Pages
        \item Sidebars
        \item Titles
        \item Subtitles (ex. Recommended Feeds)
        \item Article Tiles
        \item etc.
    \end{itemize}
    More Information for how to create common components can be found at\\
    \url{https://help.figma.com/hc/en-us/articles/360038662654-Guide-to-Components-in-Figma}\\
    \textbf{Ideas for a possible solution}
    \begin{itemize}
        \item Look at design guidelines for fonts, separation, etc. (should comply with it)
        \item Possibly copy some ideas from mockups you can find on the internet: ex.
        \url{https://www.behance.net/gallery/104152447/Mobile-App-Design-Challenge-Figma-Free-UXUI-2} \\
    \end{itemize}
    \textbf{Acceptance Criteria}
    \begin{itemize}
        \item The components above have common components in figma
        \item Components are in a seperate page in figma
        (see \url{https://help.figma.com/hc/en-us/articles/360038511293-Create-and-Manage-Pages}) \\
    \end{itemize}
\end{quotation}
Bearbeitet von Sabrina Ladner.
Reviewed von Amos Groß.

\subsubsection{SKIOS-78 Subscription Page}
Das Ticket \enquote{SKIOS-78 Subscription Page} hatte folgende User Story:
\begin{quotation}
    As a user of Skiosa I would like to have a page for my subscriptions so that I can use them.

    Goal of this story is to use the backend components provided in SKIOS-77 and to create a subscription page following our design.
    Note: Subscription indicator of mockup will not be in scope for this story. \\

\textbf{Acceptance Criteria}
\begin{itemize}
    \item Subcription page follows design of figma mockups
    \item Page contains list of the current users subscriptions
    \item For non logged in users, this page redirects to login
\end{itemize}
\end{quotation}
Bearbeitet von Amos Groß.
Reviewed von Lukas Huida.

\subsubsection{SKIOS-94 Overview Page}
Das Ticket \enquote{SKIOS-94 Overview Page} hatte folgende User Story:
\begin{quotation}
    As a user I would like to have a homepage.

    Goal of this story is to use the endpoints of the preliminary recomendation service and to display them as seen in the mockup.
    The recomended feeds are not to be regarded here. \\

\textbf{Acceptance Criteria}
\begin{itemize}
    \item Overview Page exists at root ('/')
    \item Overview Page follows design scheme 
\end{itemize}
\end{quotation}
Bearbeitet von Amos Groß.
Reviewed von Simon Morgenstern.

\subsubsection{SKIOS-114 Roadmap in Confluence}
Das Ticket \enquote{SKIOS-114 Roadmap in Confluence} hatte folgende User Story:
\begin{quotation}
    As a Product owner I would like a roadmap in confluence. \\
\textbf{Acceptance Criteria}
\begin{itemize}
    \item Preliminary Roadmap in Front Page of Confluence is Replaced
    \item Roadmap reflects or PIs
\end{itemize}
\end{quotation}
Bearbeitet von Amos Groß.
Reviewed von Lukas Huida.

\subsubsection{SKIOS-123 Latex Requirements Page}
Das Ticket \enquote{SKIOS-123 Latex Requirements Page} hatte folgende User Story:
\begin{quotation}
    As a team member I would like to have our requirements documented in confluence so that I won't fail this class. \\
\textbf{Acceptance Criteria}
\begin{itemize}
    \item A page is created in LaTeX
    \item Has a table explaining changes made to these over the course of the project
    \item Page includes:
    \begin{itemize}
    \item functional requirements
    \item non functional requirements 
    \item optional and mandatory
    \end{itemize}
\end{itemize}
\end{quotation}
Bearbeitet von Amos Groß.
Reviewed von Lukas Huida.

\subsubsection{SKIOS-78 Subscription Page}
Das Ticket \enquote{SKIOS-78 Subscription Page} hatte folgende User Story:
\begin{quotation}
    As a user of Skiosa I would like to have a page for my subscriptions so that I can use them.
    Goal of this story is to use the backend components provided in SKIOS-77 and to create a subscription page following our design.
    Note: Subscription indicator of mockup will not be in scope for this story. \\
\textbf{Acceptance Criteria}
\begin{itemize}
    \item Subcription page follows design of figma mockups
    \item Page contains list of the current users subscriptions
    \item For non logged in users, this page redirects to login
\end{itemize}
\end{quotation}
Bearbeitet von Amos Groß.
Reviewed von Lukas Huida.

\subsubsection{SKIOS-94 Overview Page}
Das Ticket \enquote{SKIOS-94 Overview Page} hatte folgende User Story:
\begin{quotation}
    As a user I would like to have a homepage.
    Goal of this story is to use the endpoints of the preliminary recomendation service and to display them as seen in the mockup.
    The recomended feeds are not to be regarded here. \\
\textbf{Acceptance Criteria}
\begin{itemize}
    \item Overview Page exists at root ('/')
    \item Overview Page follows design scheme 
\end{itemize}
\end{quotation}
Bearbeitet von Amos Groß.
Reviewed von Simon Morgenstern.

\subsubsection{SKIOS-114 Roadmap in Confluence}
Das Ticket \enquote{SKIOS-114 Roadmap in Confluence} hatte folgende User Story:
\begin{quotation}
    As a Product owner I would like a roadmap in confluence. \\
\textbf{Acceptance Criteria}
\begin{itemize}
    \item Preliminary Roadmap in Front Page of Confluence is Replaced
    \item Roadmap reflects or PIs
\end{itemize}
\end{quotation}
Bearbeitet von Amos Groß.
Reviewed von Lukas Huida.

\subsubsection{SKIOS-123 Latex Requirements Page}
Das Ticket \enquote{SKIOS-123 Latex Requirements Page} hatte folgende User Story:
\begin{quotation}
    As a team member I would like to have our requirements documented in confluence so that I won't fail this class. \\
\textbf{Acceptance Criteria}
\begin{itemize}
    \item A page is created in LaTeX
    \item Has a table explaining changes made to these over the course of the project
    \item Page includes:
    \begin{itemize}
    \item functional requirements
    \item non functional requirements 
    \item optional and mandatory
    \end{itemize}
\end{itemize}
\end{quotation}
Bearbeitet von Amos Groß.
Reviewed von Lukas Huida.

\subsubsection{SKIOS-128 Mock Recomendations for simmilar articles}
Das Ticket \enquote{SKIOS-128 Mock Recomendations for simmilar articles} hatte folgende User Story:
\begin{quotation}
    As a frontend developer I would like a mock endpoint for “simmilar article” recomendations.
This story is a part of recomendation service
\textbf{Acceptance Criteria}
\begin{itemize}
    \item Resolver is created in recomendation service
    \item Resolver returns a random list of articles when provided with an article ID
\end{itemize}
\end{quotation}
Bearbeitet von Jonas Eppard.
Reviewed von Amos Groß und Tim Horlacher.

\subsubsection{SKIOS-113 LaTeX Organigram/Orthography}
Das Ticket \enquote{SKIOS-113 LaTeX Organigram/Orthography} hatte folgende User Story:
\begin{quotation}
    As a team member I would like to have our orthography documented in confluence in LaTeX. \\
\textbf{Acceptance Criteria}
\begin{itemize}
    \item A page is created in LaTeX
    \item Page includes Diagram displaying team members
    \item Page includes short explanations for the jobs that the following roles undertook (even when in overlap with SCRUM)
    \begin{itemize}
        \item Service Lead
        \item Architects
        \item Product owner
        \item Scrum Master
        \item Team Member (optional)
    \end{itemize}
    \item Page is written in text form (prosa) and refers to diagram/organigramm
    \item Page explains WHY we chose to structure our team in this manner   
\end{itemize}
\end{quotation}
Bearbeitet von Theo Krinitz
Reviewed von Jannik Springer.

\subsubsection{SKIOS-109 backend mutation for feeds}
Das Ticket \enquote{SKIOS-109 backend mutation for feeds} hatte folgende User Story:
\begin{quotation}
As a developer, I would like to have muations for creating and deleting feeds.
Goal of this story is to create Graphql Mutation to add a feed into our database. \\
\textbf{Acceptance Criteria}
\begin{itemize}
    \item Graphql mutation is created for feeds
    \item calling one of the mutations results in a feed being created in the database
    \item calling the mutation is possible only knowing the url of a feed (all other “not null” fields are left blank or filled with defaults)
    \item calling the other mutation (by ID) results in the specified feed being deleted
    \item deletion should only be possible for logged in users with a specific keycloak role
    \item creation should be possible for all logged in users
\end{itemize}
\end{quotation}
Bearbeitet von Marcel Alex.
Reviewed von Lukas Huida.

\subsection{Sprint 4}
Ziel des letzten Sprints war es das \ac{PI} \enquote{User Interaction} fertig zustellen und die Abgabedokumentation, für die Abgabe, vorzubereiten.
Für den \acp{PI} bedeutetet dies das Erstellen von Likes, Bookmarks und der Feed Adding Page.
Zusätzlich war es hier ein Ziel die vergangenen Sprints, die Projektarchitektur und resultierenden Mockups zu dokumentieren

\subsection{Probleme und Verbesserungen}
In der Retro des letzten Sprints wurde erneut das Tool aus dem dritten Sprint verwendet.
Positiv ist aufgefallen, dass Reviews besser gehandhabt wurden, Deadlines im Vergleich zum restlichen Sprint und das Sprintziel gut eingehalten wurde.
Neue Action Items sind im Laufe dieses Meetings nicht entstanden, da keine weiteren Sprints vorgesehen sind.
Dennoch wurde ein generelles Fazit zum Projekt gezogen, welches in Kapitel \ref{proj_ergebnisse} nachzulesen ist.


\subsubsection{Start:}
\begin{itemize}
    \item Wir sind alle gestresst müssen aber auchsehen, das manche mehr Arbeit haben als andere.
    \item Scrum Master soll Team koordinieren und nicht PO (fragen wie, schaffen wir alles sollten nicht vom PO gefragt werden...)
    \item Kleinigkeiten im Review selber fixen, wenn das der Grund der Rückweisung ist
    \item Können wir mehr Struktur in unsere Meetings bekommen
    \item sich weniger beschweren, dass beste drauß machen und akzeptieren
    \item Beim erstellen von stories das Team miteinbinden und nicht im alleingang
\end{itemize}

\subsubsection{Stop:}
\begin{itemize}
    \item Einander anzufeinden jeder tut was er kann
    \item leute zu blamen obwohl man selbst nichts macht
    \item Frust aufzubauen
    \item Sich über viel Arbeit beschwerem wir machen alle so viel wir können. Fragen sind nicht immer Offensiv sondern wollen das Problem verstehen
\end{itemize}

\subsubsection{Continue:}
\begin{itemize}
    \item Deadlines
    \item An einem Strang ziehen
    \item Gute PR's und gute Reviews
    \item Einige liefern saubere PR's ab welche in angemessener Zeit bearbeitet werden, das kann so bleiben
    \item Arbeiten
\end{itemize}

\subsection{Bearbeitete User Storys}

\subsubsection{SKIOS-125 LaTeX Article and Feed Workflow explanation and sequence diagrams}
Das Ticket \enquote{SKIOS-125 LaTeX Article and Feed Workflow explanation and sequence diagrams} hatte folgende User Story:
\begin{quotation}
    As a team member I would like to have our article and feed workflow documented in latex so that I won't fail this class.
    This story is a part of chapter five (add a section after technical changes)\\
\textbf{Acceptance Criteria}
\begin{itemize}
    \item A page is created in LaTeX (Use Chapter 5) for this section
    \item Page includes UML sequence diagram for how:
    \begin{itemize}
        \item Articles are ingested, recommended and used in our platform
        \item Feeds are ingested, recommended and used in our platform
    \end{itemize}
    \item Documentation is crosschecked with Architects
    \item Diagrams are explained in the Page
\end{itemize}
\end{quotation}
Bearbeitet von Lukas Huida und Jannik Springer.
Reviewed von Amos Groß.

\subsubsection{SKIOS-145 Mutators and Resolvers for Bookmarks}
Das Ticket \enquote{SKIOS-145 Mutators and Resolvers for Bookmarks} hatte folgende User Story:
\begin{quotation}
    As a frontend developer I would like resolvers for bookmarks to be able to create and manage bookmarks. \\
\textbf{Acceptance Criteria}
\begin{itemize}
    \item A resolver is created for:
    \begin{itemize}
        \item Paginated fetch of all bookmarks
        \item Check if article is bookmarked (in article resolver!)
    \end{itemize}
    \item A mutation is created for:
    \begin{itemize}
        \item Adding bookmarks
        \item Removing Bookmarks
    \end{itemize}
\end{itemize}
\end{quotation}
Bearbeitet von Lukas Huida und Tim Horlacher.
Reviewed von Jonas Eppard.

\subsubsection{SKIOS-147 Angular Bookmarks Page} \label{story:147}
Das Ticket \enquote{SKIOS-147 Angular Bookmarks Page} hatte folgende User Story:
\begin{quotation}
    As a user I would like to have a page to view my bookmarks in. \\
\textbf{Acceptance Criteria}
\begin{itemize}
    \item Page is created
    \item Button in sidebar is created and links to this page
    \item Page displays Bookmarks of User as specified in mockup
    \item Search and Filter Icons that exist in the mockup aren't shown in the angular page
\end{itemize}
\end{quotation}
Bearbeitet von Lukas Huida.
Reviewed von Simon Morgenstern.

\subsubsection{SKIOS-144 Mutators and Resolvers for Likes}
Das Ticket \enquote{SKIOS-144 Mutators and Resolvers for Likes} hatte folgende User Story:
\begin{quotation}
    As a frontend developer I would like resolvers for likes to be able to create and manage likes. \\
\textbf{Acceptance Criteria}
\begin{itemize}
    \item A resolver is created for:
    \begin{itemize}
        \item Check if article is in likes of user (in article resolver!)
    \end{itemize}
    \item A mutation is created for:
    \begin{itemize}
        \item Liking article
        \item Removing Like
    \end{itemize}
\end{itemize}
\end{quotation}
Bearbeitet von Tim Horlacher
Reviewed von Lukas Huida.

\subsubsection{SKIOS-150 Feed Adding Page} \label{story:150}
Das Ticket \enquote{SKIOS-150 Feed Adding Page} hatte folgende User Story:
\begin{quotation}
    As a user I would like a page to add RSS feeds into our plattform.
\textbf{Acceptance Criteria}
\begin{itemize}
    \item Any logged in User can add feeds to our application
    \item Page looks like mockup
    \item Page is linked as a button in our sidebar
\end{itemize}
\end{quotation}
Bearbeitet von Jonas Eppard.
Reviewed von Simon Morgenstern.

\subsubsection{SKIOS-152 LaTeX Sprint Summaries}
Das Ticket \enquote{SKIOS-152 LaTeX Sprint Summaries} hatte folgende User Story:
\begin{quotation}
    As a member of this team, I would like to have each sprint explained in our Sprints chapter.
    \textbf{Goal of this story is to fill out the parts for:}
        \begin{itemize}
            \item “Ziele Sprint XY”
            \item “Ergebnisse Sprint XY”
            \item “Produktinkrement”
            \item if referenced in other sections: “Charts”
            \item “Probleme und Verbesserungen”
        \end{itemize}
        für jeden Sprint
\end{quotation}
\textbf{Acceptance Criteria}
    \begin{itemize}
        \item LaTeX Pages are created
        \item “Ziele Sprint XY” and “Produktincrement” are discussed with PO
        \item “Probleme und Verbesserungen” lists and discusses results of Retros
        \item Charts section is only created if referenced in any other part of this documentation
        \item A quick summary is created at the top of each sprint including:
        \begin{itemize}
            \item Timeline of the sprint
            \item Meetings of that sprint and dates for them (Bi weekly, weekly, review, retro, estimation meetings) => See (incomplete) meeting notes
            \item Capacity for that sprint
        \end{itemize}
        \item Page is spellchecked
    \end{itemize}
Bearbeitet von Sophie Gösch.
Review by Amos Groß.

\subsubsection{SKIOS-161 Angular Bookmarks Buttons}
Das Ticket \enquote{SKIOS-161 Angular Bookmarks Buttons} hatte folgende User Story:
\begin{quotation}
    As a user I would like a bookmarks button to be able to bookmark articles.
\textbf{Acceptance Criteria}
\begin{itemize}
    \item Bookmarks Icon is visible in our (Angular) Article Component
    \item Bookmarks Button is visible in the article page
    \item In both cases:
    \begin{itemize}
        \item Clicking on the Bookmark button bookmarks the article (aka. makes the appropriate call to our backend)
        \item For bookmarked articles the icon has a solid fill (same color as outline in mockup)
        \item Clicking the button changes the fill of the icon WITHOUT having to refresh the page
    \end{itemize}
\end{itemize}
\end{quotation}
Bearbeitet von Jonas Eppard.
Reviewed von Simon Morgenstern und Sabrina Ladner.

\subsubsection{SKIOS-164 LaTeX Meeting Summary}
Das Ticket \enquote{SKIOS-164 LaTeX Meeting Summary} hatte folgende User Story:
\begin{quotation}
    As a team member I would like a LaTeX page explaining the various meetings we organised in our project.
    This story will fill out section 2.3 \enquote{Zusammenarbeit} and If considered a bad title should be renamed to something containing the work \enquote{meetings}.
\textbf{Acceptance Criteria}
\begin{itemize}
    \item Page is created in LaTeX
    \item Page is Spell Checked in Review
    \item Page explains the following meetings:
    \begin{itemize}
        \item Estimations
        \item Bi Weeklys
        \item Dailys
        \item Planning
        \item Retro
        \item Review
        \item Roadmap Meetings (Held between PO and Tech Leads)
    \end{itemize}
    \item Explanations include:
    \begin{itemize}
        \item Frequency of the meeting (once a week, every sprint, every class, alternating such as with estimations etc.)
        \item Purpose of the meeting (why do we need this meeting?)
        \item Structure of the meeting (ex. bi weekly = daily, then general talking points and discussion)
        \item Result of the meeting (ex. Retro = Action items)
        \item If present team specifc adaptations to these meetings (ex. Daily being held once a week)
    \end{itemize}
    \item All explanations conform with SCRUM (see \url{https://scrumguides.org/docs/scrumguide/v2020/2020-Scrum-Guide-US.pdf#zoom=100})
    \item Individual Meetings are not explained in this section
\end{itemize}
\end{quotation}
Bearbeitet von Jannik Springer und Amos Groß.
Jeweils gegenseitig reviewed von Amos Groß und Jannik Springer.

\subsubsection{SKIOS-166 \LaTeX Technical Changelog}
Das Ticket \enquote{SKIOS-166 \LaTeX Technical Changelog} hatte folgende User Story:
\begin{quotation}
    As a team member, I would like changes to the technical understanding of our project to be documented.
    Goal of this story is to explain how our requirements regarding RSS, etc. changed throughout the project changed our requirements. \\
    This story will complete section 5.3 \enquote{Technische Änderungen zur anfänglichen Struktur}. \\
    Initial Requirements are not documented here (this is done in chapter 1).
\textbf{Acceptance Criteria}
\begin{itemize}
    \item Page is created in LaTeX
    \item Page is Spell Checked in Review
    \item Page lists changes to our technical understanding of the project:
    \begin{itemize}
        \item How RSS feeds work
        \item nessessity of a Recomendation Service
    \end{itemize}
    \item Page Explains changes to our requirements due to these changes
    \begin{itemize}
        \item Removing a creator page (due to lack of settings, ex. ttl)
    \end{itemize}
\end{itemize}
\end{quotation}
Bearbeitet von Tim Horlacher.
Reviewed von Amos Groß.

\subsubsection{SKIOS-165 LaTeX Technologies Summary (Technical)}
Das Ticket \enquote{SKIOS-165 LaTeX Technologies Summary (Technical)} hatte folgende User Story:
\begin{quotation}
    As a team member I would like a LaTeX page explaining the technologies used in our project.
    This will be section 2.1 \enquote{Eingesetzte Technologien}. \\
\textbf{Acceptance Criteria}
\begin{itemize}
    \item Page is created in LaTeX
    \item Page is spell checked
    \item Page explains the following (technical) technologies:
    \begin{itemize}
        \item Typescript (and Javascript)
        \item devcontainers (vscode)
        \item Keycloak
        \item GraphQL
        \item Apollo
        \item TypeORM
        \item TypeGraphql
        \item Postgresql
        \item node
        \item git
        \item k3s
        \item traefik
        \item argocd
        \item drone
    \end{itemize}
    \item Explanations for technial technologies include (in total max. one paragraph):
    \begin{itemize}
        \item quick summary 
        \item reason for us using it
        \item potential alternative considerations 
        \begin{itemize}
            \item Rest vs Graphql
            \item Postgresql vs. Mongo/Maria
            \item Keycloak vs. manual JWTs
        \end{itemize}
    \end{itemize}
\end{itemize}
\end{quotation}
Bearbeitet von Lukas Huida und Tim Horlacher.
Reviewed von Amos Groß.

\subsubsection{SKIOS-170-Latex Technologies Summary (Collaboration)}
Das Ticket \enquote{SKIOS-170 \LaTeX Technologies Summary (Collaboration)} hatte folgende User Story:
\begin{quotation}
    As a team member I would like a LaTeX page explaining the technologies used in our project.
    This will be section 2.1 \enquote{Eingesetzte Technologien}. \\
\textbf{Acceptance Criteria}
\begin{itemize}
    \item Page is created in LaTeX
    \item Page is spell checked
    \item Page explains the following (Collaboration) technologies (in total max. one paragraph):
    \begin{itemize}
        \item Github (Pull Requests)
        \item Jira
        \item Confluence
        \item Figma
    \end{itemize}
    \item Explanations of Collaboration Technologies include:
    \begin{itemize}
        \item our workflow with these (tickets, pull requests)
        \item Integrations (ex. Jira with Github)
        \item our reason for working with these technologies
    \end{itemize}
\end{itemize}
\end{quotation}
Bearbeitet von Tim Horlacher und Lukas Huida.
Reviewed von Amos Groß.

\subsubsection{SKIOS-169 Link Angular Application Up}
Das Ticket \enquote{SKIOS-169 Link Angular Application Up} hatte folgende User Story:
\begin{quotation}
    As a user I would like our applicatiion to have more links.
\textbf{Acceptance Criteria}
\begin{itemize}
    \item Link from Article to feed is created
    \item The following links work:
    \begin{itemize}
        \item Clicking on feed in subscription => feed
        \item Clicking on article in subscription => article
        \item Article in Homepage => article
        \item Article in feed overview => article
        \item feed in article view => feed
    \end{itemize}
    \item Skiosa side bar contains buttons for and links to: 
    \begin{itemize}
        \item Recommendations => '/'
        \item Subscriptions => ‘/subscriptions’
        \item Feed adding page,(see SKIOS-150 Seite~\ref{story:150}) 
        \item Bookmarks see SKIOS-147 Seite~\ref{story:147}
        \item Explore button is removed
    \end{itemize}
    \item Clicking on the Skiosa logo leads to the homepage
    \item Subscriptions, Bookmarks, Settings use login guard
\end{itemize}
\end{quotation}
Bearbeitet von Jonas Eppard.
Reviewed von Simon Morgenstern und Amos Groß.

\subsubsection{SKIOS-126 \LaTeX Frontend Pages Explanation and Activity Chart}
Das Ticket \enquote{SKIOS-126 \LaTeX Frontend Pages Explanation and Activity Chart} hatte folgende User Story:
\begin{quotation}
    As a team member I would like to have our use frontend pages documented in confluence so that I won't fail this class.
\textbf{Acceptance Criteria}
\begin{itemize}
    \item A page is created in LaTeX
    \item Page is spellchecked
    \item Page ist integrated into our Mockups page (See linked issue)
    \item Page includes Aktionsdiagramm for all pages of our application
    \item Page explains the diagrams in text form (prosa)
\end{itemize}
\end{quotation}
Bearbeitet von Sabrina Ladner.
Reviewed von Amos Groß.

\subsubsection{SKIOS-151 Latex Roadmap Chapter}
Das Ticket \enquote{SKIOS-151 Latex Roadmap Chapter} hatte folgende User Story:
\begin{quotation}
    As a member of this team I would like to have a road map chapter, so that I won't fail this class.
    Note: this story should be done by the PO \\
\textbf{Acceptance Criteria}
\begin{itemize}
    \item LaTeX Chapter is created and filled out
    \item Chapter explains PIs
    \item Chapter explains Rationale behind Planning
\end{itemize}
\end{quotation}
Bearbeitet von Amos Groß
Reviewed von Jonas Eppard.

\subsubsection{SKIOS-153 LaTeX Mockup showcase}
Das Ticket \enquote{SKIOS-153 LaTeX Mockup showcase} hatte folgende User Story:
\begin{quotation}
    As a team member I would like to have a page explaining our Mockups. \\
\textbf{Acceptance Criteria}
\begin{itemize}
    \item Page is is created in LaTeX
    \item Page is spellchecked
    \item Page explains our Design Language
    \begin{itemize}
        \item What a design language is (quick definition)
        \item What our design language is
        \item What our colorscheme is
    \end{itemize}
    \item Page explains our usage of Figma
    \begin{itemize}
        \item Common Components
        \item Common Colors (figma styles)
    \end{itemize}
    \item Page showcases every page of our figma designs
    \begin{itemize}
        \item Includes Screenshot
        \item Includes Quick summary (2 sentences on the behaviour of the page and what it shows, ex. clicking on subscribe in feed overview subscribes a user) 
        \item Focus on Actions that are not explained in use case ( section 1.4)
    \end{itemize}
\end{itemize}
\end{quotation}
Bearbeitet von Theo Krinitz.
Reviewed von Amos Groß.

\subsubsection{SKIOS 82 Feed Overview Page}
Das Ticket \enquote{SKIOS 82 Feed Overview Page} hatte folgende User Story:
\begin{quotation}
As a user of Skiosa, I would like to view a feed, so that I can decide if I want to subscribe to it.
Content of this story is to implemented an overview page for feeds based on our mockups (this will exclude recommended feeds if part of the mockup). \\
\textbf{Acceptance Criteria}
\begin{itemize}
    \item The page displays: 
    \begin{itemize}
        \item name
        \item description
        \item articles
    \end{itemize}
    \item Page is designed as specified in figma+
    \item The subscribe button subscribes a user (if not logged in redirects to login page)
\end{itemize}
\end{quotation}
Bearbeitet von Marcel Alex.
Reviewed von Lukas Huida.

\subsubsection{SKIOS-146 Angular Likes}
Das Ticket \enquote{SKIOS-146 Angular Likes} hatte folgende User Story:
\begin{quotation}
As a user I would like a likes component to be able to like articles.
Goal of this story is to implement the likes button for articles in the view page and individual article components. \\
\textbf{Acceptance Criteria}
\begin{itemize}
    \item Heart Icon is visible in our (Angular) Article Component
    \item Like Button is visible in the article page
    \item In both cases:
    \begin{itemize}
        \item Clicking on the Heart likes the article (aka. makes the appropriate call to our backend)
        \item For liked articles this heart has a solid fill (same color as outline in mockup)
        \item Clicking the heart changes the fill of the icon WITHOUT having to refresh the page
    \end{itemize}
\end{itemize}
\end{quotation}
Bearbeitet von Marcel Alex.
Reviewed von Ladner Sabrina.

\subsubsection{SKIOS-141 LaTeX Organisational Changes}
Das Ticket \enquote{SKIOS-141 LaTeX Organisational Changes} hatte folgende User Story:
\begin{quotation}
As a member of this team I would like to document all organisational changes that happened to this team.
Goal of this page is to summarise and analyse the effectiveness of our organisational changes (using charts, etc.).
This page thusly illustrates what we learnt from SWE.\\
THIS IS A VERY IMPORTANT STORY, WE NEED TO GET THIS RIGHT!\\
\textbf{Helpful resources}
\begin{itemize}
    \item Take Action Items from Retros into account
\end{itemize}
\textbf{Acceptance Criteria}
\begin{itemize}
    \item Page is created in LaTeX
    \item All text is spellchecked
    \item Page summs up organisational changes made throughout the Project and analyses their effectiveness.
    \item The Page uses charts to analyse the efectiveness wherever possible. Examples are:
    \begin{itemize}
        \item WIP chart for WIP limits (see needs review)
        \item Burnup chart for capacity estimates
    \end{itemize}
    \item At \textbf{minimum} it should include the following changes:
    \begin{itemize}
        \item Capacity Estimations
        \item Estimation Meeting moved out of the Planning Metting
        \item Stories being explained by service leads and not PO
        \item Guidelines for Code
        \item Guidelines for Issues (definitions for Acceptance Criteria, etc.)
        \item WIP limit
    \end{itemize}
    \item If no more changes than the ones named above could be found, a meeting is set up with Scrum Master and PO to clarify if this is a definitive list
\end{itemize}
\end{quotation}
Bearbeitet von Simon Morgenstern, Amos Groß, Theo Krinitz und Sabrina Ladner.
Reviewed von Amos Groß und Ladner Sabrina.

\subsubsection{SKIOS-149 User Settings Page}
Das Ticket \enquote{SKIOS-149 User Settings Page} hatte folgende User Story:
\begin{quotation}
As a user I would like a settings page so that I can change my colorscheme and visit keycloak.
Storage of preferences is done locally.

\textbf{Acceptance Criteria}
\begin{itemize}
    \item Page looks as specified in figma (excluding fields not explained below)
    \item Page includes settings for:
    \begin{itemize}
        \item Changing Colorscheme
        \item Links to keycloak
        \item Contact Information
    \end{itemize}
    \item Settings button (in sidebar) turns into login button
    \item Light/Dark Mode Button is changed to settings icon (see attachments)
    \item Going to page when not logged in redirects to keycloak
\end{itemize}
\end{quotation}
Bearbeitet von Simon Morgenstern und Jonas Eppard.
Reviewed von Lukas Huida.
