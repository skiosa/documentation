% !TeX root = ../dokumentation.tex

\chapter{Sprints}

\section{Sprint 1}
\todo{Beschreibung des Produktincrements}

\section{Ziele Sprint 1}
\todo{Ziel Sprint 1}

\section{Ergebnisse Sprint 1}

\subsection{Produktincrement}
\subsection{Charts}

\subsection{Probleme und Verbesserungen}
\todo{Retro Ergebnisse}


\subsection{Bearbeitete User Storys}
\subsubsection{SKIOS-22 POC for keycloak}
Das Ticket \enquote{SKIOS-22 POC for keycloak} hatte folgende User Story:
\begin{quotation}
    As a developer I would like to have a POC for two services using keycloak~\parencite{web/Keycloak} so that I`ll know how users will be managed in the future. \\
    \textbf{Acceptance Criteria}
    \begin{itemize}
        \item A backend service has an api that can get information about the logged in user
        \item A frontend service can create users (in any primitive way) and log in + send a request to the aformentioned endpoint
        \item Our keycloak~\parencite{git/skiosa/orm} instance is configured to work with both services
    \end{itemize}
\end{quotation}
Bearbeitet von Tim Horlacher.
Reviewed von Lukas Huida.

\subsubsection{SKIOS-29 Undesigned Angular Boilerplate}
Das Ticket \enquote{SKIOS-29 Undesigned Angular Boilerplate} hatte folgende User Story:
\begin{quotation}
    As a developer I would like a boilerplate for angular to be created so that I can start work on the project. \\
    \textbf{Acceptance Criteria}
    \begin{itemize}
        \item An angular boilderplate exists with basic pages and routing
        \item A central structure for requests (using services) is created
        \item Infrastructure for passing global data (such as a username) is implemented
    \end{itemize}
\end{quotation}
Bearbeitet von Tim Horlacher.
Reviewed von Lukas Huida.

\subsubsection{SKIOS-31 Evaluate Core-Service}
Das Ticket \enquote{SKIOS-31 Evaluate Core-Service} hatte folgende User Story:
\begin{quotation}
    As a developer I would like to know which ORM will be used and whether or not to base our APIs around GraphQL~\parencite{web/GraphQL} or to stick with REST within the Core service. \\
    \textbf{Acceptance Criteria}
    \begin{itemize}
        \item It is clear, what ORM to use and is documented somewhere (ex. comments of this story)
        \item It is clear if GraphQL~\parencite{web/GraphQL} is used and findings are documented somewhere aswell
    \end{itemize}
\end{quotation}
Bearbeitet von Tim Horlacher.
Reviewed von Lukas Huida.

\subsubsection{SKIOS-34 List of Participants}
Das Ticket \enquote{SKIOS-34 List of Participants} hatte folgende User Story:
\begin{quotation}
    As our lecturer I would like a list of group members so that I can grade the project. \\
    \textbf{Acceptance Criteria}
    \begin{itemize}
        \item A list is created including all 11 names and matnr.
        \item Said list is given to the professor
    \end{itemize}
\end{quotation}
Bearbeitet von Tim Horlacher.
Reviewed von Lukas Huida.

\subsubsection{SKIOS-52 Devcontainer Guide}
Das Ticket \enquote{SKIOS-52 Devcontainer Guide} hatte folgende User Story:
\begin{quotation}
    As a developer I would like a guide on the usage of devcontainers for development. \\
    \textbf{Acceptance Criteria}
    \begin{itemize}
        \item A guide is created in confluence
        \item A sample devcontainer is implemented
        \item Other developers are taught about devcontainers 
    \end{itemize}
\end{quotation}
Bearbeitet von Tim Horlacher.
Reviewed von Lukas Huida.

\subsubsection{SKIOS-13 Create Basic Infrastructure}
Das Ticket \enquote{SKIOS-13 Create Basic Infrastructure} hatte folgende User Story:
\begin{quotation}
    Als Product Owner möchte ich die Möglichkeit haben das unser Produkt weltweit sicher erreichbar ist. Um das zu erreichen, muss ein Server mit einer passenden gemietet werden. \\
    \textbf{Acceptance Criteria}
    \begin{itemize}
        \item Die Domain \enquote{Skiosa.de} bestellt und dem Server zugewiesen.
        \item Einen passenden Server bestellt und mit notwendigen Tools wie SSH aufgesetzt.
        \item K3S Basis Installation.
        \item kubectl von Remote erreichbar und eingerichtet.
        \item Firewall Regeln angelegt um nur Web, SSH und Kubectl zuzulassen.
    \end{itemize} 
\end{quotation}
Bearbeitet von Lukas Huida.
Reviewed von Tim Horlacher.

\subsubsection{SKIOS-18 Basic Kubernetes Infra}
Das Ticket \enquote{SKIOS-18 Basic Kubernetes Infra} hatte folgende User Story:
\begin{quotation}
    As a infra member I want a basic Infrastructure based on Kubernetes. This Infrastructure should container some CI/CD pipeline software and examples to implement
    a CI/CD pipeline for our services. \\
    \textbf{Acceptance Criteria}
    \begin{itemize}
        \item Working Ingress-Controller with correct Certs
        \item Self-Hosted Docker-Registry
        \item ArgoCD Setup
        \item Drone.io Setup
        \item SonarQube Setup
        \item Keycloak Setup
        \item HashiCorp Vault Setup (Optional)
    \end{itemize}
\end{quotation}
Bearbeitet von Lukas Huida.
Reviewed von Tim Horlacher.

\subsubsection{SKIOS-9 Create Continous Integration Pipeline}
Das Ticket \enquote{SKIOS-9 Create Continuos Integration Pipeline} hatte folgende User Story:
\begin{quotation}
    As a developer I want my code to be tested and analyzed by sonar after every new commit so that our platform can ensure stability. \\
    \textbf{Acceptance Criteria}
    \begin{itemize}
        \item A pipeline exists for every service.
        \item A commit pipeline is triggered after every commit on a non master branch. This pipeline runs through: unit tests and sonarqube.
        \item A pr pipeline is triggered after a pull requests is created. This pipeline runs through: unit tests, integration tests and sonarqube.
        \item A main pipeline is triggered after a merge of a master branch. This pipeline runs through: unit tests, container build, CD-Script and sonarqube.
    \end{itemize}
\end{quotation}
Bearbeitet von Lukas Huida.
Reviewed von Tim Horlacher.

\subsubsection{SKIOS-10 Create Continous Delivery Pipeline}
Das Ticket \enquote{SKIOS-10 Create Continuos Delivery Pipeline} hatte folgende User Story:
\begin{quotation}
    As a developer I want my code to be delivered to prod after I merge it to master. \\
    \textbf{Acceptance Criteria}
    \begin{itemize}
        \item pipeline is triggered after merge to master
        \item pipeline tests code with unit/sonar before merge (is already managed by ci-pipeline)
        \item pipeline tests code with newman after merge → fail leads to rollback
    \end{itemize}
\end{quotation}
Bearbeitet von Lukas Huida.
Reviewed von Jonas Eppard.

\subsubsection{SKIOS-21 Initialize Keycloak}
Das Ticket \enquote{SKIOS-21 Initialize Keycloak} hatte folgende User Story:
\begin{quotation}
    As a developer I would like to have Keycloak setup so that we can test its usefulness. \\
    \textbf{Acceptance Criteria}
    \begin{itemize}
        \item A Keycloak server is running in our cluster and is reachable.
    \end{itemize}
\end{quotation}
Bearbeitet von Lukas Huida.
Reviewed von Tim Horlacher.

\subsubsection{SKIOS-30 Preliminary Core service}
Das Ticket \enquote{SKIOS-30 Preliminary Core service} hatte folgende User Story:
\begin{quotation}
    As a developer I want to have a basic core service with some boilerplate and mock endpoints to get started in the frontend. \\
    \textbf{Acceptance Criteria}
    \begin{itemize}
        \item be able to fetch mock articles with graphql
    \end{itemize}
\end{quotation}
Bearbeitet von Lukas Huida.
Reviewed von Tim Horlacher.

\subsubsection{SKIOS-37 Decision of Programming Database}
Das Ticket \enquote{SKIOS-37 Decision of Programming Database} hatte folgende User Story:
\begin{quotation}
    As a developer I need to know which database is used, to know which database connector for ORMs are needed. \\
    \textbf{Acceptance Criteria}
    \begin{itemize}
        \item A decision of a database is done
        \item The decision is documented in Confluence
    \end{itemize}
\end{quotation}
Bearbeitet von Lukas Huida.
Reviewed von Jonas Eppard.

\subsubsection{SKIOS-38 Contribution Guidelines}
Das Ticket \enquote{SKIOS-38 Contribution Guidelines} hatte folgende User Story:
\begin{quotation}
    As a developer I would like to have a guidline for making contributions so that we have a defined process to get code resulting from stories into production. \\
\textbf{Acceptance Criteria}
\begin{itemize}
    \item Contribution Guidlines are documented in Confluence
    \item Git Guidelines for naming schemes for git commits, branch names etc., squash/rebase merging are documented
    \item Review and Testing has a process, defining what should be reviewed (integration tests, code style, etc) and who should review it
    \item Code comments and code styles are defined and agreed upon by every member
    \item Documentation style, including what to document where (Confluence vs. Git) is clarified
    \item It is clarified when a task counts as done (review of dev, review of po, dev and tech lead, only review of tech/service lead)
\end{itemize}
\end{quotation}
Bearbeitet von Simon Morgenstern.
Reviewed von Amos Groß.

\subsubsection{SKIOS-39 Guidelines Testing}
Das Ticket \enquote{SKIOS-39 Guidelines Testing} hatte folgende User Story:
\begin{quotation}
    As a developer I would like to have a guideline for testing so that I can know if I'm testing my code correctly. \\
\textbf{Acceptance Criteria}
\begin{itemize}
    \item The types of tests are clarified (unit, integration, manual, cypress/selenium, etc.)
    \item The amount of test that have to be written is clear (ex. test coverage, integration test for ever use case or error, etc)
    \item A decision is made when, where and by whom tests have to be written (ex. in separate stories per epic, implicitly with every story, etc.)
    \item All findings are documented in confluence
\end{itemize}
\end{quotation}
Bearbeitet von Simon Morgenstern.
Reviewed von Amos Groß.

\subsubsection{SKIOS-40 Testing Frameworks}
Das Ticket \enquote{SKIOS-40 Testing Frameworks} hatte folgende User Story:
\begin{quotation}
    As a developer, I would like to know which testing frameworks I need to use so that I can write tests. \\
\textbf{Acceptance Criteria}
\begin{itemize}
    \item Testing frameworks for every language are decided upon
    \item The testing frameworks are added to the services documentation in confluence
    \item Programs to to integration tests with are clear
\end{itemize}
\end{quotation}
Bearbeitet von Simon Morgenstern.
Reviewed von Amos Groß.

\subsubsection{SKIOS-49 Create Example Service}
Das Ticket \enquote{SKIOS-49 Create Example Service} hatte folgende User Story:
\begin{quotation}
    As a developer I want an example service to test the CI/CD pipeline. \\
    \textbf{Acceptance Criteria}
    \begin{itemize}
        \item An example service exists
        \item A example deployment is created
        \item A example CI pipeline is working with the example service
    \end{itemize}
\end{quotation}
Bearbeitet von Lukas Huida.
Reviewed von Tim Horlacher.

\subsubsection{SKIOS-50 Deploy Database}
Das Ticket \enquote{SKIOS-50 Deploy Database} hatte folgende User Story:
\begin{quotation}
    As a developer I want a central database to store my objects and articles. \\
    \textbf{Acceptance Criteria}
    \begin{itemize}
        \item PostgreSQL is deployed in a db namespace
        \item db is reachable in the default namespace with credentials stored at the vault
    \end{itemize}
\end{quotation}
Bearbeitet von Lukas Huida.
Reviewed von Jonas Eppard.

\subsubsection{SKIOS-64 Guideline for Issues}
Das Ticket \enquote{SKIOS-64 Guideline for Issues} hatte folgende User Story:
\begin{quotation}
    As a product owner, I would like to have a guideline for new issues so that the team can understand what their work is. \\
\textbf{Acceptance Criteria}
\begin{itemize}
    \item Issue Guideline is documented in confluence
    \item Structure for future stories is clear
\end{itemize}
\end{quotation}
Bearbeitet von Simon Morgenstern.
Reviewed von Amos Groß.

\section{Sprint 2}
\subsection{Produktincrement}
\subsection{Charts}
\subsection{Probleme und Verbesserungen}

\subsection{Bearbeitete User Storys}

\subsubsection{SKIOS-67 Enhanced Git/Jira Automation}
Das Ticket \enquote{SKIOS-67 Enhanced Git/Jira Automation} hatte folgende User Story:
\begin{quotation}
    As a developer I would like enhanced automation for Jira with the following requirements
    \begin{itemize}
        \item Dev starts new branch for his work, story changes status to → in progress
        \item Dev starts PR for his Story → story changes status to → needs review
        \item PR is merged → story changes status to → done
        \item Sync reviewer between Jira and Git (optional) 
    \end{itemize}

    \textbf{Acceptance Criteria}
    \begin{itemize}
        \item Requirements above are covered
        \item functionality works for all git-repositories
    \end{itemize}
\end{quotation}
Bearbeitet von Lukas Huida.
Reviewed von Tim Horlacher.

\subsubsection{SKIOS-69 Preliminary ORM}
Das Ticket \enquote{SKIOS-69 Preliminary ORM} hatte folgende User Story:
\begin{quotation}
    Als Programmierer will ich eine Datendefinition in einer Datenbank
    haben, und diese als Packet einbinden können.
\end{quotation}
Diese wurde folgendermaßen gelöst:
\begin{quotation}
Die Datendefinition wurde durch TypeORM~\parencite{web/TypeORM} dargestellt.
Diese wurde in dem Repository skiosa/orm~\parencite{git/skiosa/orm} als NPM-Package~\parencite{web/npm} abgebildet.
Es wurden die Article von RSS-Feeds in dem Schema aus Abbildung~\ref{fig:databaseORM} abgebildet.
\begin{figure}
    \includegraphics[width=\linewidth]{Database_Model.png}
    \caption{Datendefinition innerhalb der Datenbank}
    \label{fig:databaseORM}
\end{figure}
\end{quotation}
Bearbeitet von Jonas Eppard und Tim Horlacher.

\subsubsection{SKIOS-92 Initialize Keycloak}
Das Ticket \enquote{SKIOS-92 Initialize Keycloak} hatte folgende User Story:
\begin{quotation}
    As a developer I want to connect keycloak~\parencite{web/Keycloak} to our services and protect certain resources. \\
    \textbf{Acceptance Criteria}
    \begin{itemize}
        \item Two keycloak realms exist (production, testing)
        \item Our frontend connects to keycloak and resources are protectable
        \item Our backend connects to keycloak and resources are protectable
        \item Both realms have roles for users and admins
    \end{itemize}
\end{quotation}
Bearbeitet von Tim Horlacher.
Reviewed von Lukas Huida.

\subsubsection{SKIOS-93 Document Submission Guideline}
Das Ticket \enquote{SKIOS-93 Document Submission Guideline} hatte folgende User Story:
\begin{quotation}
    As a team member i would like to know what should be documented in our final documentation. \\
    \textbf{Acceptance Criteria}
    \begin{itemize}
        \item A confluence page should document what should be documented
    \end{itemize}
\end{quotation}
Bearbeitet von Tim Horlacher.
Reviewed von Amos Groß.

\subsubsection{SKIOS-74 Extend workflow to include wontfix}
Das Ticket \enquote{SKIOS-74 Extend workflow to include wontfix} hatte folgende User Story:
\begin{quotation}
    As a team member, I would like a wontfix state to disregard tickets.
    Goal of this story is to modify the current issue workflow used by jira.
    Note: this might require admin rights (contact PO or Architect for this) \\
    \textbf{Acceptance Criteria}
    \begin{itemize}
        \item Wontfix status exists
        \item The status contains the done property (is striked out)
    \end{itemize}
\end{quotation}
Bearbeitet von Lukas Huida.
Reviewed von Amos Groß.

\subsubsection{SKIOS-85 Repeated Polling runs}
Das Ticket \enquote{SKIOS-85 Repeated Polling runs} hatte folgende User Story:
\begin{quotation}
    As a developer I would like the polling service to be run as a cron job. \\
    \textbf{Acceptance Criteria}
    \begin{itemize}
        \item Polling service is called in Intervalls.
        \item Pod is not running all the time.
    \end{itemize}
\end{quotation}
Bearbeitet von Lukas Huida.
Reviewed von Tim Horlacher.

\section{Sprint 3}

\subsection{Produktincrement}
\subsection{Charts}
\subsection{Probleme und Verbesserungen}

\subsection{Bearbeitete User Storys}
\subsubsection{SKIOS-116 Structure and table of contents for submission (\LaTeX)}
Das Ticket \enquote{SKIOS-116 Structure and table of contents for submission (\LaTeX)}
hatte folgende User Story:
\begin{quotation}
    As a team member, I would like to have a rough structure to orient myself while writing our submission documentation.\\
    For this story, please read the requirements and guidelines set out by Garidis and develop a rough idea on how to structure our \LaTeX project.\\
    \textbf{Acceptance Criteria}
    \begin{itemize}
        \item Table of contents is created (with \textbackslash{}section, \textbackslash{}subsection, etc.) in \LaTeX
        \item Structure reflects guidelines of Garidis
        \item Structure is explained in confluence page
        \item Existing \LaTeX~stories have a defined place where their pages will go
    \end{itemize}
\end{quotation}
Dies wurde folgendermaßen gelöst:
\begin{quotation}
    Es wurde die Struktur dieses \LaTeX-Dokuments angelegt. Hierbei musste nur das Inhaltsverzeichnis
    angelegt werden, da das \LaTeX-Template schon vorhanden war.
    Die Verwendung wurde in Confluence dokumentiert.
\end{quotation}
Bearbeitet von Jonas Eppard.

\subsubsection{SKIOS-73 Place new logo in git, jira, confluence, etc}
Das Ticket \enquote{SKIOS-73 Place new logo in git, Jira, confluence, etc} hatte folgende User Story:
\begin{quotation}
    As a Product Owner I want our logo to be placed in the git repository, Jira \& confluence. \\
    \textbf{Requirements}
    \begin{itemize}
        \item Add branding for Skiosa
        \item Possibly also change look of Jira (at the top) and make it use our colorscheme.  
    \end{itemize}   
    
    \textbf{Acceptance Criteria}
    \begin{itemize}
        \item Skiosa Brandig is present in Git and Jira.
    \end{itemize}
\end{quotation}
Bearbeitet von Lukas Huida.
Reviewed von Jonas Eppard.

\subsubsection{SKIOS-76 Automatic Reviewer Suggestions}
Das Ticket \enquote{SKIOS-76 Automatic Reviewer Suggestions} hatte folgende User Story:
\begin{quotation}
    As a developer I want to have a way to automatically suggest reviewers for a PR. \\
    \textbf{Acceptance Criteria}
    \begin{itemize}
        \item Added Codeowners to every repo.
    \end{itemize}
\end{quotation}
Bearbeitet von Lukas Huida.
Reviewed von Tim Horlacher.

\subsubsection{SKIOS-91 Checklist for PRs}
Das Ticket \enquote{SKIOS-91 Checklist for PRs} hatte folgende User Story:
\begin{quotation}
    As a developer, I would like to have a reminder to check off the checklist from our contributing guidelines. \\
    \textbf{Acceptance Criteria}
    \begin{itemize}
        \item Every PR contains comment or default text including the checklist.
    \end{itemize}
\end{quotation}
Bearbeitet von Lukas Huida.
Reviewed von Jonas Eppard.

\subsubsection{}
Das Ticket \enquote{SKIOS-111 Angular Design Boilerplate} hatte folgende User Story:
\begin{quotation}
    As a frontend developer I would like our frontend to have the basic components.
    Goal of this story ist to create components for:
    \begin{itemize}
        \item Buttons
        \item The Sidebar
        \item feeds
    \end{itemize}
\textbf{Acceptance Criteria}
\begin{itemize}
    \item Angular components are created for Buttons, Sidebar, Feeds
    \item The components are visibly identical to those in figma
    \item Button components can execute code when clicked on
    \item components should look according to current color mode (light / dark)
    \item Favicon of frontend is Skiosa Favicon
    \item Sidebar expands on mobile (see hamburger icon on pages like github, etc.)
\end{itemize}
\end{quotation}
Bearbeitet von Simon Morgenstern.
Reviewed von Lukas Huida.

\subsubsection{SKIOS-119 Frontend README}
Das Ticket \enquote{SKIOS-119 Frontend README} hatte folgende User Story:
\begin{quotation}
    As a developer, I would like a README in frontend service so that I read up on information on how to use it. \\
\textbf{Acceptance Criteria}
\begin{itemize}
    \item README is created
    \item README includes:
    \begin{itemize}
        \item development guide (how to start application, use devcontainer)
        \item description of service
        \item requirements for service
        \item service lead
    \end{itemize}
\end{itemize}
\end{quotation}
Bearbeitet von Lukas Huida.
Reviewed von Amos Groß.

\subsubsection{SKIOS-120 Deployment README}
Das Ticket \enquote{SKIOS-120 Deployment README} hatte folgende User Story:
\begin{quotation}
    As a developer, I would like a README in deployment so that I read up on information on how to use it. \\
\textbf{Acceptance Criteria}
\begin{itemize}
    \item README is created
    \item README includes:
    \begin{itemize}
        \item development guide (how to start application, use devcontainer)
        \item description of service
        \item requirements for service
        \item service lead
    \end{itemize}
\end{itemize}
\end{quotation}
Bearbeitet von Lukas Huida.
Reviewed von Amos Groß.

\subsubsection{SKIOS-137 Angular Service Framework}
Das Ticket \enquote{SKIOS-137 Angular Service Framework} hatte folgende User Story:
\begin{quotation}
    As a developer I would like to have angular services for our qraphql endpoints. \\
\textbf{Acceptance Criteria}
\begin{itemize}
    \item A library is found to automatically generate or facilitate graphql query creation
    \item Framework for creating graphql queries is created
    \item existing services are modified
    \item at minimum, one service is created to explain how to use this framework
\end{itemize}
\end{quotation}
Bearbeitet von Lukas Huida.
Reviewed von Amos Groß.

\subsubsection{SKIOS-138 Use Keycloak in GraphQL}
Das Ticket \enquote{SKIOS-138 Use Keycloak in GraphQL} hatte folgende User Story:
\begin{quotation}
    As a developer I want to secure some GraphQL resolver with our Keycloak. \\
    \textbf{Acceptance Criteria}
    \begin{itemize}
        \item Every resolver can be secured with Keycloak.
        \item An example resolver is created with Keycloak authentication.
    \end{itemize}
\end{quotation}
Bearbeitet von Lukas Huida.
Reviewed von Tim Horlacher.

\subsubsection{SKIOS-123 LaTeX Requirements Page}
Das Ticket \enquote{SKIOS-123 LaTeX Requirements Page} hatte folgende User Story:
\begin{quotation}
    As a team member I would like to have our requirements documented in confluence so that I won't fail this class. \\
\textbf{Acceptance Criteria}
\begin{itemize}
    \item A page is created in LaTeX
    \item Page includes:
    \begin{itemize}
        \item functional requirements
        \item non functional requirements 
        \item optional and mandatory
    \end{itemize}
    \item Has a table explaining changes made to these over the course of the project
\end{itemize}
\end{quotation}
Bearbeitet von Amos Groß.
Reviewed von Lukas Huida.

\subsubsection{SKIOS-133 ORM Readme}
Das Ticket \enquote{SKIOS-133 ORM Readme} hatte folgende User Story:
\begin{quotation}
    As a developer, I would like a README for ORM so that I read up on information on how to use it. \\
\textbf{Acceptance Criteria}
\begin{itemize}
    \item README is created
    \item README includes:
    \begin{itemize}
        \item development guide (how to create tags and to embed it to package.json)
        \item description of what this module does
        \item code owners 
    \end{itemize}
\end{itemize}
\end{quotation}
Bearbeitet von Jonas Eppard.
Reviewed von Lukas Huida.

\subsubsection{SKIOS-122 LaTeX Architecture Overview}
Das Ticket \enquote{SKIOS-122 LaTeX Architecture Overview} hatte folgende User Story:
\begin{quotation}
    As a team member I would like to have our architecture documented in confluence so that I won’t fail this class.
\textbf{Acceptance Criteria}
\begin{itemize}
    \item A page is created in our LaTeX documentation
    \item Services are listed and explained
    \item CI/CD infrasturture is explained
    \item Security Infrastructure (Keycloak)
    \item The Page explains the architecture in text form (prosa) 
    \item A diagram (not UML) is created to visualize our plattform architecture
\end{itemize}
\end{quotation}
Bearbeitet von Tim Horlacher und Lukas Huida.
Reviewed von Amos Groß.

\subsubsection{SKIOS-84 Initial Uniform Polling Implementation}
Das Ticket \enquote{SKIOS-84 Initial Uniform Polling Implementation} hatte folgende User Story:
\begin{quotation}
    As a developer I would like an initial implementaton of the polling service.\\
    This implementation should parse a single \enquote{type} of RSS feeds including at minimum the following:
    
    \begin{itemize}
        \item Link
        \item Titel
        \item Beschreibung
    \end{itemize}

    If an article doesn't contain this information it is ignored.\\
    If an article contains more information than this that information will also be ignored (will be parsed in later story)\\
    After parsing, it loads the information into the database. \\
\textbf{Acceptance Criteria}
\begin{itemize}
    \item Service can poll the structure above
    \item Polling service loads data into DB
    \item Multiple runs of the service don't produce duplicate data
\end{itemize}
\end{quotation}
Bearbeitet von Jannik Springer.
Reviewed von Lukas Huida.

\subsubsection{SKIOS-109 Backend Feed Mutation}
Das Ticket \enquote{SKIOS-109 Backend Feed Mutation} hatte folgende User Story:
\begin{quotation}
As a developer, I would like to have muations for creating and deleting feeds.
Goal of this story is to create Graphql Mutation to add a feed into our database. \\
\textbf{Acceptance Criteria}
\begin{itemize}
    \item Graphql mutation is created for feeds
    \item calling one of the mutations results in a feed being created in the database
    \item calling the mutation is possible only knowing the url of a feed (all other “not null” fields are left blank or filled with defaults)
    \item calling the other mutation (by ID) results in the specified feed being deleted
    \item deletion should only be possible for logged in users with a specific keycloak role
    \item creation should be possible for all logged in users
\end{itemize}
\end{quotation}
Bearbeitet von Marcel Alex und Tim Horlacher.
Reviewed von Lukas Huida.

\subsubsection{SKIOS-115 Core Service README}
Das Ticket \enquote{SKIOS-115 Core Service README} hatte folgende User Story:
\begin{quotation}
    As a developer, I would like a README in core service so that I read up on information on how to use it. \\
\textbf{Acceptance Criteria}
\begin{itemize}
    \item README is created
    \item README includes:
    \begin{itemize}
        \item development guide (how to start application, use devcontainer)
        \item description of what this module does
        \item requirements for service
        \item service lead
    \end{itemize}
\end{itemize}
\end{quotation}
Bearbeitet von Tim Horlacher
Reviewed von Jonas Eppard.

\subsubsection{SKIOS-117 Polling Service README}
Das Ticket \enquote{SKIOS-117 Polling Service README} hatte folgende User Story:
\begin{quotation}
    As a developer, I would like a README in polling service so that I read up on information on how to use it. \\
\textbf{Acceptance Criteria}
\begin{itemize}
    \item README is created
    \item README includes:
    \begin{itemize}
        \item development guide (how to start application, use devcontainer)
        \item description of service
        \item requirements for service
        \item service lead
    \end{itemize}
\end{itemize}
\end{quotation}
Bearbeitet von Jannik Springer.
Reviewed von Lukas Huida.

\subsubsection{SKIOS-121 PoC READMEs}
Das Ticket \enquote{SKIOS-121 PoC READMEs} hatte folgende User Story:
\begin{quotation}
    As a developer, I would like a README in all PoCs so that I read up on information on how to use it. \\
\textbf{Acceptance Criteria}
\begin{itemize}
    \item README is created
    \item README includes:
    \begin{itemize}
        \item development guide (how to start application, use devcontainer)
        \item goal of PoC
        \item link to Story of POC
        \item result of PoC
    \end{itemize}
\end{itemize}
\end{quotation}
Bearbeitet von Tim Horlacher
Reviewed von Amos Groß.

\subsubsection{SKIOS-124 \LaTeX Use cases page and UML}
Das Ticket \enquote{SKIOS-124 \LaTeX Use cases page and UML} hatte folgende User Story:
\begin{quotation}
    As a team member I would like to have our use cases documented in LaTeX so that I won't fail this class. \\
\textbf{Acceptance Criteria}
\begin{itemize}
    \item A page is created in LaTeX
    \item Page includes a use case diagram for:
    \begin{itemize}
        \item overview page
        \item settings page
        \item view page
    \end{itemize}
    \item The different use cases for SKIOSA are explained in text form (prosa) and embedd and explain the use case diagrams
    \item Page is spellchecked and contains no spelling mistakes (TEST THIS IN REVIEW)
    \item Text has been read by at least one other team member
\end{itemize}
\end{quotation}
Bearbeitet von Jonas Eppard und Sabrina Ladner.
Reviewed von Sabrina Ladner und Amos Groß.

\subsubsection{SKIOS-82 Feed Overview Page}
Das Ticket \enquote{SKIOS-82 Feed Overview Page} hatte folgende User Story:
\begin{quotation}
    As a user of Skiosa, I would like to view a feed, so that I can decide if I want to subscribe to it.
    Content of this story is to implemented an overview page for feeds based on our mockups (this will exclude recommended feeds if part of the mockup). \\
\textbf{Acceptance Criteria}
\begin{itemize}
    \item The page displays: 
    \begin{itemize}
        \item name
        \item description
        \item articles
    \end{itemize}
    \item Page is designed as specified in figma
    \item The subscribe button subscribes a user (if not logged in redirects to login page)
\end{itemize}
\end{quotation}
Bearbeitet von Marcel Alex.
Reviewed von Lukas Huida.

\subsubsection{SKIOS-80 Article View Page}
Das Ticket \enquote{SKIOS-80 Article View Page} hatte folgende User Story:
\begin{quotation}
    As a user of Skiosa, I would like to view an article.
    Content of this story is to implemented the view page based on our mockups without its references to other articles or likes/bookmarks. \\
\textbf{Acceptance Criteria}
\begin{itemize}
    \item Title, link, description, etc. are displayed 
    \item Page uses endpoints from core-sevice to achieve this
\end{itemize}
\end{quotation}
Bearbeitet von Simong Morgenstern und Jonas Eppard.
Reviewed von Lukas Huida.

\section{Sprint 4}
\subsection{Produktincrement}
\subsection{Charts}
\subsection{Probleme und Verbesserungen}

\subsection{Bearbeitete User Storys}
\subsubsection{SKIOS-145 Mutators and Resolvers for Bookmarks}
Das Ticket \enquote{SKIOS-145 Mutators and Resolvers for Bookmarks} hatte folgende User Story:
\begin{quotation}
    As a frontend developer I would like resolvers for bookmarks to be able to create and manage bookmarks. \\
\textbf{Acceptance Criteria}
\begin{itemize}
    \item A resolver is created for:
    \begin{itemize}
        \item Paginated fetch of all bookmarks
        \item Check if article is bookmarked (in article resolver!)
    \end{itemize}
    \item A mutation is created for:
    \begin{itemize}
        \item Adding bookmarks
        \item Removing Bookmarks
    \end{itemize}
\end{itemize}
\end{quotation}
Bearbeitet von Lukas Huida.
Reviewed von Jonas Eppard.
