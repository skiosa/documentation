% !TeX root = ../dokumentation.tex

\chapter{Sprints}

\section{Sprint 1}
\todo{Beschreibung des Produktincrements}

\section{Ziele Sprint 1}
\todo{Ziel Sprint 1}

\section{Ergebnisse Sprint 1}

\subsection{Produktincrement}
\subsection{Charts}

\subsection{Probleme und Verbesserungen}
\todo{Retro Ergebnisse}


\subsection{Bearbeitete User Storys}

\subsubsection{Ticket 1}

\subsubsection{Ticket 2}

\subsubsection{Skios-35 Decision of Programming Language of Recomodation Engine}
Das Ticket \enquote{Skios-35 Decision of Programming Language of Recomodation Engine} hatte folgende User Story:
\begin{quotation}
    As a developer I need to know in which Programming Language the recomondation Engine is written. Acceptance Criteria\\
    \textbf{Acceptance Criteria}
    \begin{itemize}
        \item A decision of a Programming Language is done
        \item The decision is documented in Confluence
    \end{itemize}    
\end{quotation}
Bearbeitet von Theo Krinitz

\subsubsection{Skios-36 Decision of Programming Language of Polling Service}
Das Ticket \enquote{Skios-36 Decision of Programming Language of Polling Service} hatte folgende User Story:
\begin{quotation}
    As a developer I need to know in which programming language the polling-service is written.
    \textbf{Acceptance Criteria}
    \begin{itemize}
        \item A decision of a programming language is done
        \item The decision is documented in Confluence
    \end{itemize}
\end{quotation}
Bearbeitet von Theo Krinitz

\subsubsection{SKIOS-58 Approachable Filter}
Das Ticket \enquote{SKIOS-58 Approachable Filter} hatte folgende User Story:
\begin{quotation}
    As a developer, I would like a filter for approachable tickets to be created, so that I can pull tickets with greater ease.\\
    \textbf{Acceptance Criteria}
    \begin{itemize}
        \item A filter exists at the top of the sprint board
        \item That filter filters for approachable tickets
    \end{itemize}
\end{quotation}
Bearbeitet von Theo Krinitz

\section{Sprint 2}
\subsection{Produktincrement}
\subsection{Charts}
\subsection{Probleme und Verbesserungen}


\subsection{Bearbeitete User Storys}

\subsubsection{SKIOS-69 Preliminary ORM}
Das Ticket \enquote{SKIOS-69 Preliminary ORM} hatte folgende User Story:
\begin{quotation}
    Als Programmierer will ich eine Datendefinition in einer Datenbank
    haben, und diese als Packet einbinden können.
\end{quotation}
Diese wurde folgendermaßen gelöst:
\begin{quotation}
Die Datendefinition wurde durch TypeORM~\parencite{web/TypeORM} dargestellt.
Diese wurde in dem Repository skiosa/orm~\parencite{git/skiosa/orm} als NPM-Package~\parencite{web/npm} abgebildet.
Es wurden die Article von RSS-Feeds in dem Schema aus Abbildung~\ref{fig:databaseORM} abgebildet.
\begin{figure}
    \includegraphics[width=\linewidth]{Database_Model.png}
    \caption{Datendefinition innerhalb der Datenbank}
    \label{fig:databaseORM}
\end{figure}
\end{quotation}
Bearbeitet von Jonas Eppard und Tim Horlacher.

\subsubsection{SKIOS-83 List of initial Feeds}
Das Ticket \enquote{SKIOS-83 List of initial Feeds} hatte folgende User Story:
\begin{quotation}
    As a developer I would like a list of initial feeds that we can poll before giving users the ability to add them in manually. 
    Goal of this story is to find 20-30 good (non reddit) feeds.
    \textbf{Acceptance Criteria}
    \begin{itemize}
        \item 20-30 feeds have been found and contain actual articles
        \item Structure of feeds is similar enough for polling service 
        \item The List is documented either in confluence or JSON in polling-service
    \end{itemize}
\end{quotation}
Bearbeitet von Theo Krinitz

\section{Sprint 3}

\subsection{Produktincrement}
\subsection{Charts}
\subsection{Probleme und Verbesserungen}


\subsection{Bearbeitete User Storys}
\subsubsection{SKIOS-116 Structure and table of contents for submission (\LaTeX)}
Das Ticket \enquote{SKIOS-116 Structure and table of contents for submission (\LaTeX)}
hatte folgende User Story:
\begin{quotation}
    As a team member, I would like to have a rough structure to orient myself while writing our submission documentation.\\
    For this story, please read the requirements and guidelines set out by Garidis and develop a rough idea on how to structure our \LaTeX project.\\
    \textbf{Acceptance Criteria}
    \begin{itemize}
        \item Table of contents is created (with \textbackslash{}section, \textbackslash{}subsection, etc.) in \LaTeX
        \item Structure reflects guidelines of Garidis
        \item Structure is explained in confluence page
        \item Existing \LaTeX~stories have a defined place where their pages will go
    \end{itemize}
\end{quotation}
Dies wurde folgendermaßen gelöst:
\begin{quotation}
    Es wurde die Struktur dieses \LaTeX-Dokuments angelegt. Hierbei musste nur das Inhaltsverzeichnis
    angelegt werden, da das \LaTeX-Template schon vorhanden war.
    Die Verwendung wurde in Confluence dokumentiert.
\end{quotation}
Bearbeitet von Jonas Eppard.

\subsubsection{Skios-79 Basic Article Endpoints}
Das Ticket \enquote{Skios-79 Basic Article Endpoints} hatte folgende User Story:
\begin{quotation}
    As a front-end developer I would like to have endpoints to base article views off of so that i can create their pages.\\
    \textbf{Acceptance Criteria}
    \begin{itemize}
        \item Endpoints are created to cover the cases above
        \item This functionality is implemented in core-service
    \end{itemize}
\end{quotation}
Bearbeitet von Theo Krinitz

\subsubsection{SKIOS-134 Fix core-service tests}
Das Ticket \enquote{SKIOS-134 Fix core-service tests} hatte folgende User Story:
\begin{quotation}
    As a developer I want a working testing framework, which implements unit and integration tests.\\
    \textbf{Acceptance Criteria}
    \begin{itemize}
        \item newman is installed inside dev container
        \item npm script is added that runs newman tests
    \end{itemize}
\end{quotation}
Bearbeitet von Theo Krinitz