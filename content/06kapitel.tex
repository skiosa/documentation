% !TeX root = ../dokumentation.tex

\chapter{Projekt Ergebnisse}

\section{Erreichte Ziele}

\section{Analyse der Organisatorischen Änderungen}
\todo{Änderungen des Ablaufes erläutern und analysieren}

\subsection{Leitlinie für gemeinsame Arbeit am Quellcode}
Um die Zusammenarbeit zu vereinheitlichen wurde dafür zu Beginn des Projekts eine Leitfaden entwickelt (siehe \ref{contrib}).
Dieser umfasst wie der Code dokumentiert werden soll, Regeln für die Arbeit mit dem Versionskontrollsystem Git 
und wie Code anderer überprüft (\enquote{reviewed}) werden soll. Mit Hilfe der Leitlinie wurde eine Checkliste entwickelt 
welche von jedem Entwickler ausgefüllt werden muss, wenn er seinen Code zur Überprüfung freigibt. Richtig
eingesetzt konnte eine aktive Umsetzung der Leitlinie die Entwicklungsgeschwindigkeit vorantreiben. Die 
Übereinstimmung auf eine solche Richtlinie hatte dennoch auch Nachteile. Teilweise dauerte die Überprüfung von 
anderem Code zu lang, was wiederum abhängige Tickets und somit den Entwicklungsprozess blockierte.

\section{Personenbeiträge}
(Zusammenfassung genau beschrieben in den jeweiligen Tickets der Sprints)
