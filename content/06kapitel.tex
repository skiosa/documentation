% !TeX root = ../dokumentation.tex

\chapter{Projekt Ergebnisse}

\section{Erreichte Ziele}

\section{Analyse der Organisatorischen Änderungen}
\todo{Änderungen des Ablaufes erläutern und analysieren}
\section{Personenbeiträge}
(Zusammenfassung genau beschrieben in den jeweiligen Tickets der Sprints)

\subsection{Ausgliederung der Kapazitätsschätzungen aus den Planungsbesprechungen}

Innerhalb des Ersten Sprintes haben wir sehr viel Zeit für die Kapazitätschätzung benötigt. 
Dies hatte zur Folge, dass im Sprint Planning kaum Zeit für Produktive Diskussionen übrig war und mit laufender Dauer die Konzentration für Estimations abnahm.
Daher haben wir beschlossen das Estimation Meeting vom Sprint Planning zu entkoppeln. So kamen zu den den Bi-Weeklys und ungenutzten Vorlesungsterminen das neue Estimation Meeting hinzu.
Darüber hinaus waren die Schätzungen nicht genau genug, bzw. die Konzentrantionskapazitäten ausgeschöpft.
Dies führte dazu, dass im Planning Meeting nun genügend Zeit war und die Qualität der Estimations zunahm, was sich in den Burndown-Charts reflektiert.