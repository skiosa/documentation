% !TeX root = ../dokumentation.tex

\chapter{Projekt Ergebnisse}

\section{Erreichte Ziele}

\section{Analyse der Organisatorischen Änderungen}
\todo{Änderungen des Ablaufes erläutern und analysieren}

\subsection{Leitlinie für das Erstellen von Tickets}
In den anfänglichen Sprints ist dem Entwicklerteam aufgefallen das die Tickets allein nicht ausreichen 
um zu verstehen was erwartet wird. Um das Erstellen von Tickets zu vereinheitlichen und um sicherzustellen 
das der Entwickler notwendige information erhält wurde eine Leitlinie entwickelt. 
Diese beschreibt die grobe Einteilung eines Tickets in eine Einleitung, nützliche Informationen für die 
Bearbeitung und Akzeptanzkriterien. Die einzelnen Abschnitte enthalten zudem eine Beschreibung was diese 
enthalten sollen und was sie explizit nicht enthalten sollen. Die Grobstruktur wurde in Jira als Vorlage 
für neue Tickets eingefügt.
Die Leitlinie wurde aktiv umgesetzt und konnte dazu Beitragen das die Entwickler weniger Zeit damit 
verbringen mussten die Erwartungen des Tickets zu recherchieren. 

\section{Personenbeiträge}
(Zusammenfassung genau beschrieben in den jeweiligen Tickets der Sprints)
